Passivated steel can act as fit substrate for thin-film photovoltaics.
A thin zirconium oxide layer was applied via blade coating onto a steel foil substrate with the goal of getting a homogeneous and insulating layer.
Layers were qualitatively characterized with SEM and XRD and quantitatively characterized via current-voltage curves.
The process variables (solution concentration, number of coating layers, coating speed, coating temperature, calcination speed and calcination temperature) were optimized by particle swarm optimization (PSO) algorithm in combination with multivariate adaptive regression splines (MARS).
A correlation between the calcination temperature and the electrical properties of the ceramic layers has been revealed. 
The MARS model performed well compared to 
linear regression, kernel ridge regression and support vector regression.

%\vspace{2em}
\clearpage

\chapter*{Deutschsprachiges Abstract}
\noindent Passivierter Stahl kann als geeignetes Substrat für Dünnschicht-Photovoltaik fungieren. 
Eine dünne Schicht aus Zirkoniumoxid wurde mittels Rakelbeschichtung auf ein Stahlfolien-Substrat aufgebracht, mit dem Ziel, eine homogene und isolierende Schicht zu erhalten. 
Die Schichten wurden qualitativ mit SEM und XRD und quantitativ über Strom-Spannungs-Kurven charakterisiert. 
Die Prozessvariablen (Precursorlösungskonzentration, Anzahl der Schichten, Beschichtungsgeschwindigkeit, Beschichtungstemperatur, Kalzinationsgeschwindigkeit und Kalzinationstemperatur) wurden mithilfe des Partikelschwarmoptimierungs (PSO)-Algorithmus in Kombination mit multivariaten adaptiven Regression Splines (MARS) optimiert. Eine Korrelation zwischen der Kalzinationstemperatur und den elektrischen Eigenschaften der keramischen Schichten wurde festgestellt. Das MARS-Modell schnitt im Vergleich zu linearer Regression, Kernel-Ridge-Regression und Support Vector Regression gut ab.
