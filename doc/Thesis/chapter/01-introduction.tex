%%%%%%%%%
%%%%%%%%%%%%%%%%%%%%%%%%%%%
%%%%%%%%%%%%%%%%%%%%%%%%%%%%%%%%%%%%%%%%%%%%%%%%%%%%%%
Photovoltaics (PV) is one viable option when becoming carbon neutral. 
Furthermore, 
it uses the sun's energy directly in contrast to other energy sources (e.g. wind, water or even carbon based) and therefore 
it is fit to be used in energy harvesting projects like futuristic Dyson spheres\cite{dyson1960search} which harness the whole power output of the sun.
%
One sort of PV are CIGS (copper indium gallium sulfide) cells\cite{Vasekar2010}. 
Due to their large absorption coefficient, less material is needed and they can be made thinner and flexible. 
%\td{was sind vorteile?}
In order to make a module, multiple cells are operated in series. 
The cells must be applied to a non-conducting surface.
Glass is a good non-conducting substrate, but very rigid and brittle. 
An alternative is steel, which is ductile, inexpensive and highly available, but conducting. 
An insulating layer must therefore be applied to the steel substrate before any CIGS cells can be synthesised on top.
Polymere would be a obvious choice if not for its low melting point which makes high prodction temperatures inpractical.
A non-toxic material which is suitable for the insulation is \ch{ZrO2}. 
An economic and scalable method is doctor blading via a sol-gel process. 
Sol-gel processes often produce porous layers. 
In this work a dense, insulating and homogeneous layer is pursued. 
Machine learning can help to uncover complex non-linear relations, such as the 
dependence of the thickness and resistance of the resulting layer on production factors.
%\td{influence} of the production factors on the thickness and resistance of the resulting layer.
The minimization of the conductance is performed with a particle swarm optimization 
algorithm. 
%The resulting optimum is compared with other optimization methods.

\ds{
This work used a rather unconventional approach. 
Firstly, a problem was posed. Then, some minor literature research was done and the problem was tackled. 
Finally, the engineering problem was solved within the limits of acceptability, but not aussreichend for the author. 
The larger bulk of literature research was done after the practical work has been finished in order to solve the secondary scientific problem of predicting outputs and approximate dependencies. 
}

The remainder of this work is organized with chapter 
%\ref{sec:aims} providing insight into the motivation and the goals. Section \ref{sec:theoretical} gives
\ref{sec:theoretical} giving 
some background information on topics of PV, material science and computational science which were used in this work.
In chapter \ref{sec:exp} and \ref{sec:comp} the experimental and computational procedures are described. 
Section \ref{sec:results} is split into three sections: material specific results are presented in section \ref{sec:res-mat}, results regarding particle swarm optimization in section \ref{sec:res-emma} and further analysis in section \ref{sec:res-post-emma}.
Finally, chapter \ref{sec:outlook} summarizes and discusses the outlook and next steps.

