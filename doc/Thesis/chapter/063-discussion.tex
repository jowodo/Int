\section{Hindsight is easier than foresight}
\label{sec:res-discussion}
%\section{Discussion}
%\todo{search i.e. and replace with i.e.\ }
%\todo{go to experimental and material (IV) discussion}
%\td{I-V: 2 terminal measurement one terminal was varibale and the other the ground from $-5*10^{-3}$ V to $5*10^{-3}$ V with steps of $10^{-2}$ measure current from back bone to backbone if shorted, then can measure actual resistance of layer. resistance of steel is neglectable. In order to get an impression of the quality of the layer mutliple contact (picture) are sputtered (throuhg a mask) and statistics: Two angabewerte: the weighted durchschnit and the number of pinholes, ie the number of contacts which were shorted, have an resistance below an threshold. if tunneling, iv follows powerlaw, if direct contact, iv follows linear }
%\todo{go to anova discussion}
%- \td{The disadvantage of ANOVA is that information is lost because independent variables are assumed categorical even though they are ordinal}
%\todo{british or US?: zation vs zation} 
%\todo{ include last phase into previous section}
This section will discuss mistakes, improvements and lessons learned from each phase of this project. % in chronological order. 
Firstly,
the "\nameref{sec:phase1}" will be discussed in subsection \ref{sec:phase1}, 
followed by the "\nameref{sec:phase2}" reviewd in subsection \ref{sec:phase2} 
%, thereupon the 
and finally the "\nameref{sec:phase3}" examined in subsection \ref{sec:phase3}.
%\ref{sec:phase3} \textit{\nameref{sec:phase4}}%
% will be discussed.
%%%%%%%%%%%%%%%%%%%%%%%%%%%%%%%%%%%%%%%%%%%%%%%%%%%%%%%%%%%%%%%%%%%%%%%%%%%%%%%%%%%%%%%%%
%%%%%%%%%%%%%%%%%%%%%%%%%%%%%%%%%%%%%%%%%%%%%%%%%%%%%%%%%%%%%%%%%%%%%%%%%%%%%%%%%%%%%%%%%
%%%%%%%%%%%%%%%%%%%%%%%%%%%%%%%%%%%%%%%%%%%%%%%%%%%%%%%%%%%%%%%%%%%%%%%%%%%%%%%%%%%%%%%%%
\subsection{Research, exploration and design phase}
\label{sec:phase1}
%\td{freewriting}: 
\textbf{Scrutinize everything!} At the beginning of a project it is important 
to get a ``feel''. 
I took the first recipe as only true recipe and tried to tune it to perfection. 
The problem was that the base line of that recipe was so low such that 
improving the recipe still lead to poor performance.
%no matter how much I improved it, it was still bad. 
This was only noticed after trying out another recipe. 
The author's tip here is to read as much and mostly diverse literature as possible and then try to mimic what ever can be. 
This will 
make it possible to acquire 
%get you a 
``feel'' of what works and what does not. 
%Thinking back to some design classes I took with \href{http://www.tomtorsneyweir.com/}{Thomas Torsney-Weir} for data visualization: 
This is inspired by a quote from \href{http://www.tomtorsneyweir.com/}{Thomas Torsney-Weir}'s 
data visualization classes: ``create as many and different designs as possible''.
One should check what works, collect (dis-)advantages of methods and combine the best approaches. 
This can save you a lot of time. 

Moreover, it is important to define the objective of the project as clear and unambiguous as possible. 
Again, try to reformulate the objective in multiple ways and choose the most fitting. 
Answer which question should be answered, and establish hypotheses with corresponding null hypotheses. 
A general experimental design procedure was stated by Cherkassky and Mulier in  ``Learning from Data''\cite{cherkassky1998learning}:
\begin{enumerate}
    \item {State the problem}
    \item {Formulate the hypothesis}
    \item {Design the experiment/data generation process}
    \item {Collect the data and perform preprocessing}
    \item {Estimate the model}
    \item {Interpret the model/draw the conclusion}
\end{enumerate}
%%%%%%%%%%%%%%%%%%%%%%%%%%%%%%%%%%%%%%%%%%%%%%%%%%%%%%%%%%%%%%%%%%%%%%%%%%%%%%%%%%%%%%%%55
%%%%%%%%%%%%%%%%%%%%%%%%%%%%%%%%%%%%%%%%%%%%%%%%%%%%%%%%%%%%%%%%%%%%%%%%%%%%%%%%%%%%%%%%%
%%%%%%%%%%%%%%%%%%%%%%%%%%%%%%%%%%%%%%%%%%%%%%%%%%%%%%%%%%%%%%%%%%%%%%%%%%%%%%%%%%%%%%%%%
\subsection{Narrowing, pre-optimization and constraint phase}
\label{sec:phase2}
The selection of input and output variables, data encoding/representation and incorporating a prior knowledge into the design of the learning system is often more critical for an overall success than the design of the learning machine itself.\cite{cherkassky1998learning} %(page 25)
After having decided on a base recipe, it is important to define limits and decrease the number of independent variables. 
The later is more important due to the curse of dimensionality\cite{cherkassky1998learning}, 
which states that with increasing dimensionality it get exponentially more difficult to arrive at the same density of data points. 
By excluding input variables from the model, 
the complexity of the model can be drastically reduced. 
This leads to more information which can be extracted from the remaining data, 
be it in more detail of the dependence or higher confidence interval. 
A good measure to compare with literature and other methods is the \gls{epv}. 
The \gls{epv} in this work is round 5,
which is low in comparison with literature consulted
\cite{
ahmed2017preparation,
fernandes2010multi,
kaczmarowski2015genetic,
mahdavi2017hardness,
soltanali2014neural,
panwar2014design,
schubert2008design,
shanaghi2013experimental}.
Reducing variables is of course no easy task.
How does one decide which variables to keep and which to incorporate into the model and which to discard? 
There is an intuitive (expert) way and a mathematical way.
An expert would decide which variables to discard by his expert knowledge about the domain. 
Mathematical ways include principal component analysis and step-wise regression.
%
Additionally, having wide limits can be an advantage when the data is noisy and the dependence is not too complicated, because by moving samples to the boundaries of the design space, the effect of the noise is reduced\cite{giunta2003overview}. 
This suggest variable reduction favorable over limit narrowing.
%"By moving the samples to the boundaries of the design space, the effect of the random term is reduced\cite{giunta2003overview}."
%
There are two opposing principles which should be kept in mind when reducing variables, though:
By introducing a lot of variables into a model presupposes that they and their interactions (!) have a statistically significant effect\cite{gunst2009fractional}.
On the other hand a model can be used to tell if a variable exerts an influence\cite{haertler2014statistisch}. 
%
%One regime where overfitting occurs is when there are to many parameters to be fitted with not enough data. 
Another reason to avoid too many variables in a model is that they introduce more parameters to fit and overfitting happens in the regime where to many parameters are fitted with not enough data.
%todo{Overfitting happens in the regime where to many parameters are fitted with not enough data. }

Another step/deed which is well placed in this phase is the estimate the error by reproducing a single sample a certain number of times. 
%Where the number will then be higher than during optimization 
%\todo{include curse of dimensionality.}
%
%%%%%%%%%%%%%%%%%%%%%%%%%%%%%%%%%%%%%%%%%%%%%%%%%%%%%%%%%%%%%%%%%%%%%%%%%%%%%%%%%%%%%%%%55
%%%%%%%%%%%%%%%%%%%%%%%%%%%%%%%%%%%%%%%%%%%%%%%%%%%%%%%%%%%%%%%%%%%%%%%%%%%%%%%%%%%%%%%%%
%%%%%%%%%%%%%%%%%%%%%%%%%%%%%%%%%%%%%%%%%%%%%%%%%%%%%%%%%%%%%%%%%%%%%%%%%%%%%%%%%%%%%%%%%
\subsection{Optimization, modelling and data generation phase}
\label{sec:phase3}
During the optimization the main problems were too many in- and output variables, reproducibility and regularization. 
%The first problem 
Variable selection has already been discussed in the previous section. 
When working with stochastic algorithms it is especially important to introduce reproducibility via a seed, which determines all subsequent pseudo-random numbers. 
Finally, during the \gls{emma} optimization the input data were not regularized, %es wurde auf regularization verzichtet, 
because the regularization would be different for different generations.
Regularizing with the mean and standard deviation would be either inconsistent across generations or over overvalue the distribution of the initial generation. 
%because the distribution of 
%
%alter previous measurements, which alread beitragen to the model and the influence the decision of the next sample. 
These were the reasons for not including regularization during \gls{emma} optimization. 
On the other hand the input data could have been regularized to only take values $[-1;1]$ by defining the regularized varible as 
$x'= \frac{x-\bar{x}}{x_{max} - \bar{x}}$ with $\bar{x}= x_{\text{max}} - x_{\text{min}}$.
%On the other hand haette man auf jeden fall die input variablen normieren können. 
%Normally, for standardization the mean is substracted and it is devided by the standard deviation, which impplies a distribution of a random variable. 
%One could have assumed a standard uniform distribution. 
%%%%%%%%%%%%%%%%%%%%%%%%%%%%%%%%%%%%%%%%%%%%%%%%%%%%%%%%%%%%%%%%%%%%%%%%%%%%%%%%%%%%%%%%55
\iffalse
\begin{itemize}
    \item regularization
    \item reproducability (random in code and repeat sample)
    \item main problems too many in and output vars
%\td{two main problems: 1. too many input variables; 2. too many output varibales; at2: as some input variables were also output variables, MARS couldnt find correct BFs. This could be used though: use input var also as output var if you think they are inportant, this way they will be more likely to be choosen in basis function}
\end{itemize}
\fi
%%%%%%%%%%%%%%%%%%%%%%%%%%%%%%%%%%%%%%%%%%%%%%%%%%%%%%%%%%%%%%%%%%%%%%%%%%%%%%%%%%%%%%%%55
%%%%%%%%%%%%%%%%%%%%%%%%%%%%%%%%%%%%%%%%%%%%%%%%%%%%%%%%%%%%%%%%%%%%%%%%%%%%%%%%%%%%%%%%%
%%%%%%%%%%%%%%%%%%%%%%%%%%%%%%%%%%%%%%%%%%%%%%%%%%%%%%%%%%%%%%%%%%%%%%%%%%%%%%%%%%%%%%%%%
\iffalse
\subsection{Evaluation, interpretation and outlook phase}
\label{sec:phase4}
\begin{itemize}
    \item why lowest $\gamma$ for lowest $T_{cal}$? according which eq? emma? 
    \item why $\gamma ~ \lambda$? (minimum for layer count?) according which eq? emma? 
    \item reasons for large error per sample: ageing of solution, varying humidity, varying room temperature?, impurities in solutions, inhomogeneity of film, measureing error, systematical/thinking error? what else? 
%What is the reason for aging (O2,H2O?, not sealing accelerates) ? 
%Why does \gls{ipo} decelerate the aging process? ( ipo fits better to solvent(buoh)?)
%the original solvent was 2-methoxy-ethonanol instead of \ch{buoh}.
%\td{COMMENT: solution age was not accounted for}
    \item \td{maybe the lowering of G and p are not because of better process variables but because of me getting better at creating the layers *sad* :0 } 
\end{itemize}
\fi

%%%%%%%%%%%%%%%%%%%%%%%%%%%%%%%%%%%%%%%%%%%%%%%%%%%%%%%%%%%%%%%%%%%%%%%%%%%%%%%%%%%%%%%%5
%%%%%%%%%%%%%%%%%%%%%%%%%%%%%%%%%%%%%%%%%%%%%%%%%%%%%%%%%%%%%%%%%%%%%%%%%%%%%%%%%%%%%%%%5
%%%%%%%%%%%%%%%%%%%%%%%%%%%%%%%%%%%%%%%%%%%%%%%%%%%%%%%%%%%%%%%%%%%%%%%%%%%%%%%%%%%%%%%%5
%%%%%%%%%%%%%%%%%%%%%%%%%%%%%%%%%%%%%%%%%%%%%%%%%%%%%%%%%%%%%%%%%%%%%%%%%%%%%%%%%%%%%%%%%%%%%
%%%%%%%%%%%%%%%%%%%%%%%%%%%%%%%%%%%%%%%%%%%%%%%%%%%%%%%%%%%
%%%%%%%%%%%%%%%%%%%%%%%%%%%%%%%%%%%%%%%%%%%%%%%%%%%%%%%%%%%
%%%%%%%%%%%%%%%%%%%%%%%%%%%%%%%%%%%%%%%%%%%%%%%%%%%%%%%%%%%
%%%%%%%%%%%%%%%%%%%%%%%%%%%%%%%%%%%%%%%%%%%%%%%%%%%%%%%%%%%
%%%%%%%%%%%%%%%%%%%%%%%%%%%%%%%%%%%%%%%%%%%%%%%%%%%%%%%%%%%%%%%%%%%%%%%%%%%%%%%%%%%%%%%%5
%%%%%%%%%%%%%%%%%%%%%%%%%%%%%%%%%%%%%%%%%%%%%%%%%%%%%%%%%%%%%%%%%%%%%%%%%%%%%%%%%%%%%%%%5
%%%%%%%%%%%%%%%%%%%%%%%%%%%%%%%%%%%%%%%%%%%%%%%%%%%%%%%%%%%%%%%%%%%%%%%%%%%%%%%%%%%%%%%%5
%%%%%%%%%%%%%%%%%%%%%%%%%%%%%%%%%%%%%%%%%%%%%%%%%%%%%%%%%%%%%%%%%%%%%%%%%%%%%%%%%%%%%%%%%%%%%

%Issues of methods for learning from data: 
%\begin{itemize}
%	\item How to incorporate a priori assumptions into learning? 
%	\item How to measure model complexity (i.e. flexibility to fit the training data)?
%	\item How to find a optimal balance between the data and a priori knowledge? 
%\end{itemize}
%\cite{cherkassky1998learning}
%"Learning is the process of estimating an unkown (input,output) dependency of structure of a system using a limit number of observations."\cite{cherkassky1998learning}
%"Do not attempt to solve a specified problem by indirectly solving a harder general problem as an intermediate step."\cite{cherkassky1998learning}(p33)
%"This approach works well only when the number of training samples is large relative to the (prespecified) model complexity (or the number of free parameters)".\cite{cherkassky1998learning}(p41)
%Modeling bias in statistics is the discrepancy of the mismatch between parametric assumptions and the true dependency.
%Modeling bias is overcome by using very flexible approximation function, with the tradeoff of more complex inductive (learning) steps. 
%Any learning process requires the following(p40): 
%\begin{itemize}
%	\item A (wide, flexible) set of approximating functions $\mathbf{f}(\mathbf{x},\omega),\; \omega \in \Omega$
%	\item A priori knowledge to impose contrains
%	\item An inductive principle (what needs to be done)
%		\begin{itemize}
%			\item penalizeation (regularization) inductive principle
%			\item early stopping rules (with ANN, difficult to control and interpret)
%			\item structural risk minimization (SRM) (order according to complexity)
%			\item Bayesian Inference (add subjectivity to learning machine)
%			\item Minimum Description Length (MDL, cryptic complicated uninteresting)
%		\end{itemize}
%	\item A learning method (how does it need to be done,implementation)
%\end{itemize}
%\cite{cherkassky1998learning}
%am wichtigsten is das Ziel der untersuchung! \cite{haertler2014statistisch}
%Er könnte nun die Unterschiede zwischen den Schulen "statisteisch messen"un den Schuleinfluss in der abschliessenden Analyse "herausrechnen". \cite{haertler2014statistisch}
%%%%%%%%%%%%%%%%%%%%%%%%%%%%%%%%%%%%%%%%%%%%%%%%%%%%%%%%%%%%%%%%%%%%%%%%%%%%%%%%%%%%%%%%%%%%%
%\subsection{LINKS}
%\begin{itemize}
%    \item stat vs ML \url{https://medium.com/source-institute/ai-vs-statistics-c2485f9df126} and 
%    \item https://towardsdatascience.com/no-machine-learning-is-not-just-glorified-statistics-26d3952234e3?gi=3f94b919de45
%    \item https://towardsdatascience.com/are-you-aware-how-difficult-your-regression-problem-is-b7dae830652b calculate error/smoothness etc
%\end{itemize}
%%%%%%%%%%%%%%%%%%%%%%%%%%%%%%%%%%%%%%%%%%%%%%%%%%%%%%%%%%%%%%%%%%%%%%%%%%%%%%%%%%%%%%%%%%%%%
%%%%%%%%%%%%%%%%%%%%%%%%%%%%%%%%%%%%%%%%%%%%%%%%%%%%%%%%%%%%%%%%%%%%%%%%%%%%%%%%%%%%%%%%55
\iffalse
\begin{itemize}
    \item The most time consuming part was definitively the prelaminary studies. this time could have been verkürzt by testing a wide array of diverse recipes from the literature
    \item minimize prelaminary study time by testing a wide range of designs. scrutinize (hinterfragen) everything!
    \item \textbf{planning} gives chance of zooming out and force oneself to get an overview and to decide how much time on each task
    \item \textbf{research} phase is most important, dig into basics, discuss ideas with colleagues or go into detail about plan. Understand details of methods. Make a dummy run. Collect (dis-)advantages of different methods. Like in design project: do as many and as diverse solutions as possbile. Compare and combine the best parts. Reading will let learn from others mistakes. \\ \td{longer literature research (first solution,second solution) and PSO wrongly eingeschaetzt, could have been verhindert durch lesen the docu and more papers more thoughoughly. Ich war geblendet von dem was ich wollte und dadurch nicht realistisch}
    \item \textbf{objective} should be defined as clear as possible. It will help to set a clear goal for each todo, e.g. the primary goal of a screening experiment is to identify the active factors.\cite{miller2001using} Which question should be answered and aufstellen a null Hypothesis. This is often very trivial, but let's you focus on the correct path. 
    \item what was goal of incorporating ml into this work? 
%\td{The goal of introducing \gls{ml} to this project was to apply and delve/vertiefen into subject which I've been studying during my undergrad courses. This seemed like the perfect opputrunity. And indeed I learned a lot. Even if only although \gls{ml} is a hot topic at the moment, it can not solve every problem. Two important things have I learned regarding \gls{ml}: 
\end{itemize}
General experimental procedure from Cherkassky and Mulier Learning from Data\cite{cherkassky1998learning}:
\begin{enumerate}
    \item \textbf{State the problem}
    \item \textbf{Formulate the hypothesis}
    \item \textbf{Design the experiment/data generation process}
    \item \textbf{Collect the data and perform preprocessing}
    \item \textbf{Estimate the model}
    \item \textbf{Interpret the model/draw the conclusion}
\end{enumerate}
\iffalse
\textbf{1. Statement of the Problem} There was no clear statement of the problem. Now I would formulate it: "How to produce most insulating layer with least energy? How do we define most insulating?" That's why there are two output variables rather than one.
\textbf{2. Hypothesis Formulation} formulate an unknown dependency and define input and output variables.
\textbf{3. Data Generation and Experiment Design} can be either in control of the modeler (designed experiment) or in an observational setting. The data collection can affect the sampling distribution and influence the next steps. 
\textbf{4. Collect the Data and Perform Preprocessing} here outliers are detected and data preprocessing/encoding/feature selection. Scaling by standard deviation might be a good idea, but independent scaling of variables can lead to suboptimal representation for the learning task. Feature selection: A small number of informative features make the task of estimating dependencies easier. 
\textbf{5. Model Estimation} The main goal is to construct a model for accurate prediction of future outputs
\textbf{6. Interpret the Model and Drawing Conclusions} The interpretability and accuracy of the model are compete. In classical statistics such as linearly parametrized function will suit both requirements. More complex and flexible models might lead to better estimates with less interpretability. Identifying the most important input variables. \cite{cherkassky1998learning}
\fi
\fi
%%%%%%%%%%%%%%%%%%%%%%%%%%%%%%%%%%%%%%%%%%%%%%%%%%%%%%%%%%%%%%%%%%%%%%%%%%%%%%%%%%%%%%%%%%%%
%\td{This section is about 3. DAta Generation and Experiment Design from cherkessky's list. }
\iffalse
\begin{itemize}
    \item Dimension reduction bietet sich stark an bei solcher Datenlage. in pre-optimization!
    \item \textbf{keep it simple}: include what you already know and reduce model size to minimum, such that \gls{epv} is high.  Low \gls{epv} leads to low statistical significance, especially if the error is realtively large. (confere orig papers)
%\td{In example of documentation 2 input variables and 1-2 responses initial population=10
%, subsequent population=5, but 10 iterations. Clearly more input varibles and less 
%iterations (as in actual experiments) lead to suboptimal results.}
    \item \textbf{pre-optimization}{The primary goal of a screening experiment is to identify the active factors. A secondary goal is to provide a simple model that captures the essential features of the relationship between these active factors and the response—that is, to identify the active effects. \cite{miller2001using}} But $2^6=64$ are too many for preliminary test so maybe check base and from there do $2\cdot6=12$ experiments. Then focus on selected input vars, which might have mutliple maxima \\ \td{pre optimization was used to check if limits where okay, but could also have been used to sieve out factors without impact on response (see miller 2001 \cite{miller2001using} section 1) }
    \item \textbf{factorial design} (classical \gls{doe}) is more robust at feature extraction for error laden data \cite{giunta2003overview} putting data on the extrems makes trend extraction easier. "By moving the samples to the boundaries of the design space, the effect of the random error terms is reduced."
    \item only include relevant variables\cite{gunst2009fractional}, but include all variables which could contribute\cite{haertler2014statistisch}
%-\td{"Similarly, it is often wise not to plan a comprehensive experiment that involves a large number of factors of interest. Such an experiment presupposes that most or all of the factors are important contributors to changes in the response variable and that they contribute jointly; i.e., that individual factor effects and their interactions are statistically significant and meaningful." - from Gunst2009}
%but on the other hand: If one ain't sure a if factor is relevant, the model should be able to detect if there is a influence. - from Haertler2014?
    \item which input variables could have been left out? why? justify with intuition and maths
    \item selection of in- and output vars is more important than coice of learning algorithm. 
%The process informal part of selection of input and output variables, data encoding/representation and incorporating a prior knowledge into the design of the learning system is often more critical for an overall success than the design of the learning machine itself.\cite{cherkassky1998learning} (page 25)
    \item discuss epv and compare with lit: when searching for appropriate algo, came across multi papers \todo{where should be discussed?}
%Ahmed 2017		generations 50  popu 20 2in 1out \cite{ahmed2017preparation}
%Arabs 1994		about varying pop size
%Fernandes 2010		MOGA 500gen 1000pop \cite{fernandes2010multi}
%Hosseini 2016		GA 50 pop 1000 generations 3in 1out 
%Kaczmarowski2015	GA 50 pop 50-100 generation data calculated \cite{kaczmarowski2015genetic}
%Madhavi 2017		GA+ANN	46 data	18 hiddenlayers init pop 30, pop 100 % 5in 1out \cite{mahdavi2017hardness}
%Soltananali 2014	GA+ANN	100 Generations 85 data points 6 input 8 hiden 1 output data from literature \cite{soltanali2014neural}
%Panwar 2014		GA+Simulation	4 parameters    % 2in2out? \cite{panwar2014design}
%schubert 2008		GA+simulation \cite{schubert2008design,
%Shanaghi 2013 		MOGA (GA+ANN) population size 100 \cite{shanaghi2013experimental}
    \item Eine gefährliche Fehlerquelle ist dei Erwartung eines bestimmten Ergebnisses durch den Experimentator. \cite{haertler2014statistisch}
        %%%%%%%%%%%%%%%%%%%%%%%%%%%%%%%%%%%%%%%%%%%%%%%%%%%%%%%%
    \item curse of dimensionality
%- "Je grösser nämlich die Anzahl freier Parameter im Modell ist, umso grösser ist der erforderliche Datenumfang und umso ungenauer wir die statistisch gewonnene Aussgae bei gleichem Datenumfang." - Gisela Härtler\cite{haertler2014statistisch}
%\td{"An important concept in inferential statistics is that the amount of information you can learn about a population is limited by the sample size. The more you want to learn, the larger your sample size must be." - by \url{https://blog.minitab.com/en/adventures-in-statistics-2/the-danger-of-overfitting-regression-models} } 
%\td{"overfitting a regression model occurs when you attempt to estimate too many parameters from a sample that is too small." - same source}
%by \url{https://blog.minitab.com/en/adventures-in-statistics-2/the-danger-of-overfitting-regression-models} 
%\td{"However, if the effect size is small or there is high multicollinearity, you may need more observations per term." -same source} 
%by \url{https://blog.minitab.com/en/adventures-in-statistics-2/the-danger-of-overfitting-regression-models} 
\iffalse %%%%%%%%%%%%%%%%%%%%%%%%%%%%%%%%%%%%%%%%%%%%%%%%%%%%%%
\textbf{Curse and Complexity of Dimensionality}\cite{cherkassky1998learning} chapter 3.1.
goal is to estimate function with finite samples, so it's always inaccurate (biased). 
Meaning, full estimate only possible with high sampling density, which is difficult for high dimensional functions.
\begin{enumerate}
	\item Sample sizes yielding the same density increse exponentially with dimension.
	\item A large radius is needed to enclose a fraction of the data points in a high-dimensional space.
	\item Almost every point is closer to an edge than to another point.
	\item Almost every point  is an outlier in its own projection.
\end{enumerate}
\textbf{Other statistics}
\begin{itemize}
    \item f-test (include to ANOVA \url{https://quantifyinghealth.com/f-statistic-in-linear-regression/})
    \item p-value: Is the probability that a data point is observed under the null hypothesis. A small value argues against the null hypothesis. 
    \item t-test: will tell you if a variable is statistically significant. 
    \item f-test: will tell you if a group of variables is jointly significant
    \item T-test (signal-noise-ratio \href{https://statisticsbyjim.com/hypothesis-testing/t-tests-1-sample-2-sample-paired-t-tests/}{(click here)}): if variable are stat. significant; 
    \item F-test (ratio of variances \href{https://statisticsbyjim.com/anova/f-tests-anova/}{click here}): if group of vars stat. sig \url{https://quantifyinghealth.com/f-statistic-in-linear-regression/}
\end{itemize}
\fi %%%%%%%%%%%%%%%%%%%%%%%%%%%%%%%%%%%%%%%%%%%%%%%%%%%%%%%%%5
\end{itemize}
\fi


\td{when adding dep var to model this has two folgen: - beeinflusst the BF and the optimization, but bcs relatiev weight is only 0.05 for each, no big beeinflussung on optimization} 
