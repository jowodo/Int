%%%%%%%%%%%%%%%%%%%%%%%%%%%%%%%%%%%%%%%%%%%%%%%%%%%%%%%%%%%%%%%%%%%%%%%%%%%%%%%%%%%%%%%%%%%%%%%%%
%%
%%  Die vorliegenden LaTeX-Folien stehen Mitarbeiter*innen und Studierenden der Universität Wien
%%  zur Verfügung und sind ausschließlich zur Verwendung in Forschung und Lehre der Universität  
%%  Wien vorgesehen. Das Copyright der LaTeX-Vorlagen liegt bei der Universität Wien.
%%
%%  These LaTeX slides are available to employees and students of the University of Vienna 
%%  and are intended exclusively for use in research and teaching at the University of Vienna. 
%%  The copyright of the LaTeX templates is held by the University of Vienna.
%%
%%%%%%%%%%%%%%%%%%%%%%%%%%%%%%%%%%%%%%%%%%%%%%%%%%%%%%%%%%%%%%%%%%%%%%%%%%%%%%%%%%%%%%%%%%%%%%%%%


%\documentclass[hyperref={pdfpagelabels=false}, aspectratio=43, t, draft]{beamer}  %% Choose aspectratio=43 or aspectratio=169
\documentclass[hyperref={pdfpagelabels=false}, aspectratio=43, t]{beamer}  %% Choose aspectratio=43 or aspectratio=169

	
%%%%%%%%%%%%%%%%%%%%%%%%%%%%%%%%%
%% ====== Define Style ======= %% 
%%%%%%%%%%%%%%%%%%%%%%%%%%%%%%%%%

%% ======== required inputs ========

	%% ====== title page ======
	\title{Particle Swarm Optimization for Chemical Passivation of Steel Substrate with Zirconium Oxide Layers} %% Presentation Title
	\newcommand{\titleBackground}{2}    						%% Background graphic: 0 = no; 1 = yes; 2 = yes with more text space
	\newcommand{\gPath}{figures/}									%% set graphics path
	\newcommand{\graphicsTitleBackground}{title}    %% Filename of background graphic


%% ======== optional inputs ========
	
	%% ====== title page ======
	\subtitle{PSO for Steel Passivation with ZrO}  %% (optional) Subtitle, comment out to avoid include
	\newcommand{\authorText}{Johann Dorn}				%% (optional) Author
%	\newcommand{\logoTitleFooterR}{grey}  %% (optional) additional logo in title page footer, most right
%	\newcommand{\logoTitleFooterM}{grey}  %% (optional) additional logo in title page footer, more right (if horizontal spacing not adequate, save multiple logos in one file, use \logoTitleFooterR)
%	\newcommand{\logoTitleFooterL}{grey}  %% (optional) additional logo in title page footer, right (if horizontal spacing not adequate, save multiple logos in one file, use \logoTitleFooterR)

	%% ====== footer ======
	\newcommand{\textFooter}{\subtitle - \authorText} %% (optional) Text for footline, e.g. title, comment out lines to avoid include, max. 1 line
	\newcommand{\slideNumberLabelFooter}{Slide}    	%% Page/Slide/Folie/Seite
	\date{2024-10-30} %% Date, comment out to get current date

	%% ====== header ====== 
	%\newcommand{\sectionHeader}{Section (optional)} %% (optional) Text before section number, e.g. Section or Kapitel; comment out to avoid headline

	%% ====== TOC ====== 
	%\newcommand{\includeTocAtBeginSection}{}   		 %% (optional) Table of contents at begin of every section, comment out to avoid include


%% ======== load beamer style ========
	\usepackage{beamerthemeuniwien2017}
	
	
%%%%%%%%%%%%%%%%%%%%%%%%%%%%%%%%%%%%%
%% ====== Further Preamble ======= %% 
%%%%%%%%%%%%%%%%%%%%%%%%%%%%%%%%%%%%%

%% ====== settings (optional) ======
	\usenavigationsymbolstemplate{}     %% (optional) Comment out to include navigation 

\usepackage{siunitx}
\usepackage{chemformula}

%%%%%%%%%%%%%%%%%%%%%%%%%%%%%%%%%%%
%% ====== Begin Document ======= %% 
%%%%%%%%%%%%%%%%%%%%%%%%%%%%%%%%%%%

\begin{document}

%% ====== Title Page ======

\maketitle

													
%% ====== Outline at beginning of document ======

%\begin{frame}{Outline (optional)}
%	\tableofcontents							%% (optional) Table of contents, comment out lines to avoid include
%\end{frame}


%% ====== Slides ======
%TODO: - read aqua/good recipe/mars paper
%01. title slide 
	% - aufhänger
	% - AI and PV
	% - Lab or Computer 
%02. agenda
	% - more like process  from beginning to finish
%03. background info
	% - PV CIGS substrate % https://de.wikipedia.org/wiki/Chalkopyrit
	% https://en.wikipedia.org/wiki/Copper_indium_gallium_selenide_solar_cell
\begin{textFrame}
	{Copper indium gallium selenide solar cell}
	{1}{}
%	\begin{itemize}
%		\item lorem
%		\item ipsum
%		\item \dots
%	\end{itemize}
\begin{table}[tbh]
	\small
    \center
%	\iffalse
	\resizebox{\textwidth}{!}{
		\begin{tabular}{cccccc}
			\hline
			\hline
			Material&   Type&    Band Gap [eV]&    $\lambda$ [nm]&    Abs. coef. $\alpha$ [cm$^{-1}$]    &Penetration Depth [$\mu$m]\\
			\hline
			c-Si&   indirect&   1.12&   600&    \num{4000}&    2.5\\
			c-Si&   indirect&   1.12&   1000&    \num{64}&    150\\
			c-Si&   indirect&   1.12&   1100&    \num{3.5}&    290\\
			a-Si&   direct&      1.7&    600&    \num{40000}&  0.25\\
			CdTe&   direct&      1.45&    600&    \num{37000}&  0.3\\
			GaAs&   direct&      1.42&    600&    \num{40000}&  0.2\\
			\hline
			\hline
		\end{tabular}
	}
    \caption{Photonic properties of several established PV materials (\cite{mertens2020photovoltaik})}
	\label{tab:cigs:alpha}
%	\fi
%
	\resizebox{.45\textwidth}{!}{
    \begin{tabular}{cccc}
        \hline\hline
		Empirical Formula&   Band Gap [\si{eV}{}]\\
        \hline
		\ch{CuInSe2}&  1.04\\
		\ch{CuInS2}&  1.5\\
		\ch{CuGaSe2}&  1.7\\
		\ch{CuGaS2}&  1.55\\
        \hline\hline
    \end{tabular}
}
    \caption{Band gaps of different chalcopyrites (K. Mertens 2015 "Photovoltaik")}
\end{table}
\end{textFrame}

\begin{graphicsFrame}{Copper Indium Gallium Sulfide}{}{0.37}{left}{cigs}{Rau, Schock (2013) "Solar Cells"}
		\begin{itemize}
			\item Abgeschieden wohnen sie in Buchstab-hausen an der Küste des Semantik, eines großen Sprachozeans.
			\item Ein kleines Bächlein namens Duden fließt durch ihren Ort und versorgt sie mit den nötigen Regelialien.
		\end{itemize}
\end{graphicsFrame}

\begin{graphicsFrame}{CIGS modules}{}{0.1}{}{cigs_mod}{}

\end{graphicsFrame}
%04. hypothesis 
%05. manual methods
	% - precursor solution preparation 
		% Anwar 2017 https://iopscience.iop.org/article/10.1088/1757-899X/257/1/012087
		% Hu 2016 https://www.sciencedirect.com/science/article/abs/pii/S0272884216312548
\begin{frame}{test}

\begin{table}[h]
	\center
	\caption{Composition of different buthanolic solutions}
	\label{tab:rec2}
	\begin{tabular}{rlllll}
		\hline
		recipe	&1F		&2F		&3F		&4F		&5F		\\
		\hline
		\ch{BuOH} [ml]	&4.95	&4.9	&4.85	&4.8	&4.75	\\
		\ch{ZrPro} [ml]	&0.05	&0.1	&0.15	&0.2	&0.25	\\
		\ch{AcAc} [ml]	&0.0125	&0.025	&0.0375	&0.05	&0.0625	\\
		\ch{IPO}/\ch{AcOH} [ml]	&2		&2		&2		&2		&2		\\
		\hline
	\end{tabular}
\end{table}
\end{frame}

	% - doctor blading = tape castig 
	% - calcination 
\begin{frame}{Tape casting and Calcination}
	\begin{itemize}
		\item doctor blading construction 
		\item calcination infos (max temperatur, heating rate)
	\end{itemize}
\end{frame}
	% - sputtering contacts
	% - measuring current-voltage characteristics
\begin{frame}{Sputtering and I-V measurement}
	\begin{itemize}
		\item Aluminium Contact through mask
		\item Silver paste
	\end{itemize}
\end{frame}
%06. computational pipeline 
	% - PSO + MARS = EMMA
	% PSO https://doi.org/10.1109/CEC.2010.5586165
\begin{frame}{EMMA = PSO + MARS}
	\begin{itemize}
		\item Particle Swarm Opitimisation 
	\end{itemize}
\end{frame}
%07. results measuring 
	% - SEM % check which sample 
\begin{graphicsFrame}{SEM micrographs}{}{0.37}{right}{sem}{}
		\begin{itemize}
			\item (a) first recipe on FTO
			\item (b) 10 layers on steel
			\item (c) cross section of 5 layers on FTO
			\item (d) side view of scratched layer on steel
		\end{itemize}
\end{graphicsFrame}

	% - I-V % https://en.wikipedia.org/wiki/Current%E2%80%93voltage_characteristic
\begin{graphicsFrame}{I-V characteristics}{}{0.1}{}{iv1}{}\end{graphicsFrame}
\begin{graphicsFrame}{I-V characteristics}{}{0.1}{}{iv2}{}\end{graphicsFrame}
\begin{graphicsFrame}{I-V characteristics}{}{0.1}{}{iv3}{}\end{graphicsFrame}
	% - EMMA 
%08. results EMMA 
\begin{graphicsFrame}{Pin Hole Density}{}{0.1}{}{phd}{}\end{graphicsFrame}
\begin{graphicsFrame}{Leakage Current}{}{0.1}{}{leakage_w}{}\end{graphicsFrame}
\begin{graphicsFrame}{Leakage Current}{}{-1}{}{leakage}{}\end{graphicsFrame}
\begin{graphicsFrame}{Leakage Current}{}{0.24}{right}{leakage}{}\end{graphicsFrame}
\begin{graphicsFrame}{Leakage Current}{}{0.37}{right}{leakage}{}\end{graphicsFrame}
\begin{graphicsFrame}{EMMA generations}{}{0.1}{}{emma-gen}{}\end{graphicsFrame}
	% - gen 
%09. comparison with other statistics 
	% - comparing with other stats
	\begin{frame}{Comparison with Statistics}
\begin{table}[htb]
	\centering
%    \caption{Post EMMA vergleich}
	\resizebox{.95\textwidth}{!}{
	\begin{tabular}{c cc cc cc cc}
    \hline\hline
        $\gamma$&  MAE(e)&   MSE(e)&   MAE(p)&   MSE(p)&   MAE(c)&   MSE(c)&   MAE(a)& MSE(a) \\
    \hline
        MARS&   10& 171&    28& 1084&   30& 1280&   17& 548\\
        LR&    13& 250&    24& 747&    34& 1327&   17& 454\\
%        KRR& 15& 415&    26& 1048&   28& 1586&   19& 676\\
        KRR &17 &548 &30 &1477 &35 &2167 &23 &931\\
        SVM& 15&  415&    26& 1050&   28& 1588&   19& 677\\
        \hline
%    \hline\hline
	\end{tabular}
}
	\vspace{1em}
    %
	\resizebox{.95\textwidth}{!}{
	\begin{tabular}{c cc cc cc cc}
    \hline\hline
        $\rho$&  MAE(e)&   MSE(e)&   MAE(p)&   MSE(p)&   MAE(c)&   MSE(c)&   MAE(a)& MSE(a) \\
    \hline
        MARS&   0.14&   0.03&   0.41&   0.21&   0.39&   0.18&   0.25&   0.11\\
        LR&    0.19&   0.06&   0.30&   0.15&   0.38&   0.13&   0.24&   0.09\\
%        KRR& 0.30&   0.22&   0.47&   0.38&   0.52&   0.44&   0.34&   0.26\\
        KRR &0.25 &0.12 &0.38 &0.24 &0.43 &0.29 &0.31 &0.17\\
%        SVM& 0.28&   0.12&   0.37&   0.21&   0.42&   0.25&   0.32&   0.16\\
        SVM &0.23 &0.13 &0.41 &0.28 &0.45 &0.33 &0.31 &0.20\\
    \hline\hline
	\end{tabular}
}
    \caption{Comparison of MAE and MSE of different prediction methods for different data sets: EMMA data set~(e), pre-EMMA data set~(p), pre-EMMA data set within EMMA bounds~(c) and complete dataset~(a)}
	\label{tab:post-emma}
\end{table}
\end{frame}
    
%10. summary 
	% - created ZrO2
	% - optimized with EMMA
	% - double checked with stats
%11. conclusion 
%12. end 




\iffalse
\begin{graphicsFrame}{Layout ``Body with figure, small right''}{short}{0.63}{left}{graphic_rs}{\textcopyright~Universität Wien/derknopfdruecker.com}

		Random formula
		\[
			(0,1)\ni t\mapsto\frac{\partial}{\partial t} g(t,\omega)=\int_{( 0,1-t]}\frac{G(dr,\omega)}{1-r}
		\]
		Another random formula
		\begin{equation}\label{eq1}
			\int_{( G(0+,\cdot),1)}\frac{ f_{\mathcal{G},G^{\leftarrow}(t,\cdot),X}}{1-G^{\leftarrow}(t,\cdot)}\,dt
			= f_{\mathcal{G},G,X}\quad \textrm{a.s.}
		\end{equation}
		And another, even more random formula
		\[
			\mathbb{P}(X\leq Z-\varepsilon)\leq
			\mathbb{P}(X\leq q_{\mathcal{G},\delta}(X)-\varepsilon )< \delta
		\]

\end{graphicsFrame}
													

\begin{textFrame}{Layout ``Titel und Inhalt'' = Standardlayout Überschriften}{1}{Referenz, Quellen- oder Copyright-Angabe bei Bedarf einfügen}

	\begin{itemize}
		\item Fließtext 22 pt, Mustertext: Weit hinten, hinter den Wortbergen, fern der Länder Vokalien und Konsonantien leben die Blindtexte.
		\item Abgeschieden wohnen sie in Buchstabhausen an der Küste des Semantik, eines großen Sprachozeans. Ein kleines Bächlein namens Duden fließt durch ihren Ort und versorgt sie mit den nötigen Regelialien.
		\item Es ist ein paradiesmatisches Land, in dem einem gebratene Satzteile in den Mund fliegen. Nicht einmal von der allmächtigen Interpunktion werden die Blindtexte beherrscht – ein geradezu unorthographisches Leben.

	\end{itemize}
\end{textFrame}

\begin{textFrame}{Layout ``Titel und Inhalt wenig Text''}{0.7}{Referenz, Quellen- oder Copyright-Angabe bei Bedarf einfügen}

	\begin{itemize}
		\item Mustertext: Weit hinten, hinter den Wortbergen, fern der Länder Vokalien und Konsonantien leben die Blindtexte.
		\item Abgeschieden wohnen sie in Buchstabhausen an der Küste des Semantik, eines großen Sprachozeans. Ein kleines Bächlein namens Duden fließt durch ihren Ort und versorgt sie mit den nötigen Regelialien.
		\item Nicht einmal von der allmächtigen Interpunktion werden die Blindtexte beherrscht – ein geradezu unorthographisches Leben.
	\end{itemize}
\end{textFrame}


%% ====== Section  ======

\section{Layout ``Abschnittsüberschrift''}

\begin{sectionFrame}{section.jpg}{mit Untertitel}
\end{sectionFrame}


\section{Layout ``Abschnittsüberschrift ohne Bild''~-- Titel kann auch mehrzeilig sein}

\begin{sectionFrame}{}{mit Untertitel}
\end{sectionFrame}


\begin{textFrame2}{Layout ``Zwei Inhalte''}{}{
		\begin{itemize}
			\item In die Inhaltsplatzhalter können unterschiedliche Elemente eingefügt werden.
			\item Mustertext: Abgeschieden wohnen sie in Buchstabhausen an der Küste des Semantik, eines großen Sprachozeans. 
					Ein kleines Bächlein namens Duden fließt durch ihren Ort und versorgt sie mit den nötigen Regelialien.
		\end{itemize}
}{Referenz, Quellen- oder Copyright-Angabe}{}{\includegraphics[width=\linewidth]{\gPath diagram.jpg}}%
{Referenz, Quellen- oder Copyright-Angabe}
\end{textFrame2}

\begin{textFrame2}{Layout ``Vergleich''}{Vorteile des XYZ-Modells in Zusammenhang mit dem Projekt}%
{
		\begin{itemize}
			\item Weit hinten, hinter den Wortbergen, fern der Länder Vokalien und Konsonantien leben die Blindtexte.
			\item Abgeschieden wohnen sie in Buchstabhausen an der Küste des Semantik, eines großen Sprachozeans. 
		\end{itemize}
}{}{Nachteile des XYZ-Modells in Zusammenhang mit dem Projekt}{
		\begin{itemize}
			\item Abgeschieden wohnen sie in Buchstabhausen an der Küste des Semantik, eines großen Sprachozeans.
			\item Ein kleines Bächlein namens Duden fließt durch ihren Ort und versorgt sie mit den nötigen Regelialien. 
		\end{itemize}
}{}
\end{textFrame2}

\begin{graphicsFrame}{Layout ``Inhalt mit Bild \\ klein rechts''}{short}{0.6}{left}{graphic_rs}{\textcopyright~Universität Wien/derknopfdruecker.com}

		\begin{itemize}
			\item Abgeschieden wohnen sie in Buchstab-hausen an der Küste des Semantik, eines großen Sprachozeans.
			\item Ein kleines Bächlein namens Duden fließt durch ihren Ort und versorgt sie mit den nötigen Regelialien.
			\item Es ist ein paradiesmatisches Land, in dem einem gebratene Satzteile in den Mund fliegen. Abgeschieden wohnen sie in Buchstabhausen an der Küste des Semantik, eines großen Sprachozeans. 
		\end{itemize}

\end{graphicsFrame}

\begin{graphicsFrame}{Layout ``Inhalt mit Bild klein \\ links''}{short}{0.6}{right}{graphic_ls}{\textcopyright~Universität Wien/derknopfdruecker.com}

		\begin{itemize}
			\item Abgeschieden wohnen sie in Buchstab-hausen an der Küste des Semantik, eines großen Sprachozeans.
			\item Ein kleines Bächlein namens Duden fließt durch ihren Ort und versorgt sie mit den nötigen Regelialien.
			\item Es ist ein paradiesmatisches Land, in dem einem gebratene Satzteile in den Mund fliegen. Abgeschieden wohnen sie in Buchstabhausen an der Küste des Semantik, eines großen Sprachozeans.
		\end{itemize}

\end{graphicsFrame}

\begin{graphicsFrame}{Layout ``Inhalt mit Bild größer rechts''}{}{0.37}{left}{graphic_rm}{\textcopyright~Universität Wien/Barbara Mair}

		\begin{itemize}
			\item Abgeschieden wohnen sie in Buchstab-hausen an der Küste des Semantik, eines großen Sprachozeans.
			\item Ein kleines Bächlein namens Duden fließt durch ihren Ort und versorgt sie mit den nötigen Regelialien.
		\end{itemize}

\end{graphicsFrame}

\begin{graphicsFrame}{Layout ``Inhalt mit Bild größer links''}{}{0.37}{right}{graphic_lm}{\textcopyright~Universität Wien/Barbara Mair}

		\begin{itemize}
			\item Abgeschieden wohnen sie in Buchstab-hausen an der Küste des Semantik, eines großen Sprachozeans.
			\item Ein kleines Bächlein namens Duden fließt durch ihren Ort und versorgt sie mit den nötigen Regelialien.
		\end{itemize}

\end{graphicsFrame}

\begin{graphicsFrame}{Layout ``Bild groß mit Titel''}{}{0.1}{}{graphic_xl}{\textcopyright~Universität Wien/Barbara Mair}

\end{graphicsFrame}

\begin{graphicsFrame}{}{}{0.24}{right}{graphic_ll}{\textcopyright~Universität Wien/derknopfdruecker.com}

		\begin{itemize}
			\item Layout „Bild groß mit Text“ 
			\item Abgeschieden woh-nen sie in Buchstab-hausen an der Küste des Semantik, eines großen Sprachozeans.
			\item Ein kleines Bächlein namens Duden fließt durch ihren Ort und versorgt sie mit den nötigen Regelialien.
			\item Es ist ein paradiesma-tisches Land, in dem einem gebratene Satzteile in den Mund fliegen.
		\end{itemize}

\end{graphicsFrame}

\begin{graphicsFrame}{Layout ``Bild abfallend mit Titel''}{}{-1}{}{graphic_xxl}{\textcopyright~Universität Wien/derknopfdruecker.com}

\end{graphicsFrame}

\begin{graphicsFrame2}{}{0.19}{graphic_2c_lm}{\textcopyright~Universität Wien/derknopfdruecker.com}{graphic_2c_rm}{\textcopyright~Universität Wien/Barbara Mair}

		\begin{itemize}
			\item Layout „Twei Bilder mit Text rechts“ 
			\item Ein kleines Bächlein namens Duden fließt durch ihren Ort und versorgt sie mit den nötigen Regelialien.
			\item Es ist ein para-diesmatisches Land, in dem einem gebratene Satzteile in den Mund fliegen.
		\end{itemize}

\end{graphicsFrame2}

\begin{graphicsFrame2}{Layout ``Titel, zwei Bilder mit Text rechts''}{0.17}{graphic_2c_ls}{\textcopyright~Universität Wien/Barbara Mair}{graphic_2c_rs}{\textcopyright~Universität Wien/Barbara Mair}

	\textbf{Ein kleines Bächlein namens Duden} fließt durch ihren Ort und versorgt sie mit den nötigen Regelialien.
	\smallskip

	Es ist ein paradiesmatisches Land, in dem einem gebratene Satzteile in den Mund fliegen.

\end{graphicsFrame2}

\begin{graphicsFrame2}{Layout ``Titel, zwei Bilder''}{0}{graphic_2c_ll}{\textcopyright~Universität Wien/Barbara Mair}{graphic_2c_rl}{\textcopyright~Universität Wien/derknopfdruecker.com}

\end{graphicsFrame2}

\fi

%% ====== End Document ====== %%
\end{document}
