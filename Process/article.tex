\documentclass[a4paper]{article}
\usepackage[margin=3cm]{geometry}
\usepackage{xcolor}
\usepackage{graphicx}
\usepackage{lipsum}
\usepackage{ngerman}
\usepackage{pdfpages}
\usepackage{hyperref}
\usepackage{url}
\usepackage{ragged2e} % to justify text
\newcommand{\todo}[1]{\textbf{\textcolor{red}{#1}}}
\newcommand{\td}[1]{\textbf{\textcolor{red}{#1}}}
\title{Doctor Blading of ZrO$_2$}
\author{Johann Dorn}

\begin{document}
\maketitle
\iffalse
my notes
\fi
\justify


\section{Cutting of the steel foil}
There is a red foil cutter in the vacuum room, which cuts the foil without much bending.
The foil is cut into 2.5 x 2.7 mm plates.
The small plates are marked with an diamond cutter pen.
The plates are cleaned with 1ml of Hellmanex III in 50-100 ml dion. water in the sonic bath for 15 min, then in dion. water for 15 min and finally in isopropanol for 15 min. 
The samples are dry blown with dry N$_2$ and stored until doctor blading.

\section{Solution}
The recipe for the solution was adapted from ref \cite{Hu2016}.
The standard concentration will be described first and then the differences of higher concentrated solutions:
4.9 ml of 1-buthanol (BuOH) are put into a beaker glass or similar while stirring. 
0.1 ml of zirconium(IV)propoxide solution (Zr(PrO)$_4$) are added and the time is written down. 
After 10 to 15 minutes 0.05 ml (approximately one mole equivalent of Zr(PrO)$_4$) acetylactate (AcAC) is added and stirred for another 10 to 15 minutes. 
Finally, 1 ml of isopropanol (IPO) is added to the mixture and stirred for additional 20-30 minutes. 
Following stirring times (in minutes) were tested and didn't have an influence on stability of the solution: 10-10-20, 10-10-45, 30-30-180. 
In order to make a double concentrated solution, the volume of Zr(PrO)$_4$ and AcAc is doubled and BoOH is decreased by the volume of Zr(PrO)$_4$. 

\begin{table}[h]
	\centering
	\begin{tabular}{clllll}
		\hline
				&1F		&2F		&3F		&4F		&5F	\\
		\hline
		BuOH	&4.95	&4.9	&4.85	&4.8	&4.75	\\
		Zr(OPr)	&0.05	&0.1	&0.15	&0.2	&0.25	\\
		AcAc	&0.0125	&0.025	&0.0375	&0.05	&0.0625	\\
		IPO		&2		&2		&2		&2		&2		\\
		\hline
	\end{tabular}
\end{table}

\section{Doctor blading}
The temperature of the heating plate is set to 200 $^o$C.
The temperature is set and waited until reached.
The sample is placed on the vacuum plate and tested if it can be held by the under pressure.
The velocity is set and a mini test run is performed. 
The blade is put in position. 
100-125 $\mu$l of solution is applied with an 10-1000 $\mu$l pipette and the doctor blading is started immediately. 
After evaporation of the solution, the vacuum is turned of, the 'blade pusher' put into initial position, the blade removed and excess solution removed with a wipe. 
The small metal plate is transferred to the hot heating plate and rests on there for 3-5 min. 
The process is repeated as wished. 
\bibliographystyle{ieeetr}
\bibliography{fp}

\clearpage

\end{document}
