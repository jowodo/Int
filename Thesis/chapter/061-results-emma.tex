%%%%%%%%%%%%%%%%%%%%%%%%%%%%%%%%%%%%%%%%%%%%%%%%%%%%%%%%%%%
%%%%%%%%%%%%%%%%%%%%%%%%%%%%%%%%%%%%%%%%%%%%%%%%%%%%%%%%%%%
\subsection{EMMA}
\begin{itemize}
    \item what was the best result? 
    \item discuss layers: less layers are better in prediction of? 
    \item Every output var is independent of each other, so $v_{cal}$ can act as test 
heating rate was one of the dependent variables with the intention of minimizing the variable. 
It can also be used as test to see how well the EMMA performes (or rather, more precisely MARS)
It doesn't influence the fit for the other splines, but it influences the choice of samples therefore it might have slowed down the process
Overall there were too many variables involved for such a small dataset
    \item plot G vs nr or boxplot (see also \texttt{/Code/PCA/overview.gp} maybe sort -k3) 
    \item which values were actually used in EMMA and how were they calculated?
        G, phd, vCal and layers
    \item plot generation against conductivity
    \item \todo{make figure with G over time}
    \item DOE (only experiment planing)
    \item EMMA/MARS (influence of TCal) \texttt{Code/Statistics/sub/mars.R}
    \item ANOVA (influence of TCal) \texttt{Code/Statistics/sub/anova.R}
    \item lin regression (influence of conc) \texttt{Code/Statistics/sub/linreg.py}
    \item grid search 
    \item KRR (???)
    \item SVM (???)
\end{itemize}

and cite something pro forma \cite{ncbi1butanol}
\td{plot the predicted data for variables which should be excluded (Tcal, Vcal,Conc, layers)}
