\label{sec:res-post-emma}
%describe what awaits the reader in this chapter:
in this section the experimental results are analyzed with different methods and compared with the original method EMMA which was integrated into the optimization. 
%%%%%%%%%%%%%%%%%%%%%%%%%%%%%%%%%%%%%%%%%%%%%%%%%%%%%%%%%%%%%%%%%%%%%%%%%%%%%%%%%%%%%%%%5
\subsubsection{ANOVA}\label{sec:res-anova}
\textbf{EMMA}
The \gls{rf} produced by the last iteration of \gls{emma} put a high \td{significance/effect/correlation} 
on the calcination temperature $T_{cal}$ with regard to the \td{measurables} of the conductance. 
\Gls{anova} is used to double check this. 
%Now \gls{anova} can be used to check if the data 

\textbf{What can/did we test?}
With \gls{anova} we can check if the probability of arriving at these results was by pure chance or not. 
The null hypothesis is that the results were obtained by pure chance without any dependence of the insulation of a sample on the process variables.
%We tested for each input variable the influence on rho and gamma 
A one-way \gls{anova} was performed for every independent variable on $\rho$ and $\gamma$. 
%and also as a test on vcal and $\lambda$. 
We used an $\alpha$ value of 0.05 which means that 
the null hypothesis is rejected if the p-value is under 0.05. 
Usually, the $\alpha$-value is chosen to be 0.01 or 0.05\cite{hoffman2020concept,sellke2001pvalues}.
The choice of a relatively high alpha-value is motivated by the rather small amount of data and low \td{EPV}.
As mentioned by Sellke et al.\cite{sellke2001pvalues} a p-value of 0.05 might tempt 
to guess that in 1 out of 20 cases the results could be false positives, 
whereas the frequency is much higher. 
%1 out of 20 times this result could be 
%produced by pure chance given that the null hypothesis is true. 
%If the p-value is below that value, then we choose to accept, that the data is not due to
%chance but due to the influence of the independent variables on the dependent variables. 
\td{could test anova on different generations data sets} 

\textbf{What were the results?}
%\Gls{anova} shows that for the mentioned indep and dep pairs, the means divided by 
%indep groups are not equal and that there is at least one group that is different. 
For both $\rho$ and $\lambda$ (for both the \gls{emma} dataset and the extended dataset) 
the F values for $T_{cal}$ were in all cases under 0.01, 
indicating a real influence of $T_{cal}$ on the conductance.
Both datasets exhibit a p-value $< 0.05$ for the interaction $T_{cal}$:$v_{cal}$ on $\rho$.
\td{And the extended data set has p-value smaller than 0.01 of vcal on phd. 
Whereas vcal on phd for \gls{emma} dataset has p-value of slightly over 0.05.}
\td{why tcal:vcal on phd the same for both datasets? zufall}

\textbf{What do the results mean?} 
Anova bestaetigt that TCal has influence on both G and phd. 
\td{how to interpret p-values of 0.01 and 0.05? 
They mean that under the assumption that the null hypothesis is correct, 
there is a 1\% and 5\% chance of arriving at such data as observed.}
It seems that the probability of getting such data per chance that Tcal has no influence 
(null hypothesis is true) is less then 5\% (even less than 1\%). \td{NO bcs \cite{sellke2001pvalues}}
Important to mention, that anova uses categorical data for inputs, 
where our data is numerical and thus information is lost

%\textbf{Pre-writing:}
%\Gls{anova} doesn't provide a lot of extra information. 
%wrong the second is not for phd but for extended data set
%but for phd even lower than 0.1 \%
%Anova shows that the influence of TCal is significant as shown by the F 

%%%%%%%%%%%%%%%%%%%%%%%%%%%%%%%%%%%%%%%%%%%%%%%%%%%%%%%%%%%%%%%%%%%%%%%%%%%%%%%%%%%%%%%%5
\subsubsection{Linear Regression}
I thought that the influnce of $conc$ and $\lambda$ are the biggest, but they are just the smalles absolute values. 
Thus, eventhough $T_{cal}$ might have a smaller coefficient for lin reg, the influence on the dependent variable might be larger. 
The biggest influece on $\rho$ is indeed $T_{cal}$, next with about a third of the influence $T_{doc}$. 
Both $T_{cal}$ and $T_{doc}$ have positive coefficients, i.e. a negative influence on the resistance. 
Then both $\lambda$ and $v_{doc}$ both have about a fift ob the largest influence and a negative influence. 
An indication that the data is strongly tainted by error (see also section \ref{sec:res-anova}) is the low $R^2$ score of 0.41, 
which is (sad but true) even unterboten by the $R^2$ of lin fit of $G$, 0.34. 
The coefficients, though share the signs. 
This is only true if the extended data set is used and therefore again rather by chance. 
It is even more \td{erstaunlich} that EMMA managed to decrease the average of $pG$ an $\rho$ with each generation (see figure \ref{fig:emma-gen}. 
Or was is only luck? 

$v_{cal}$ and $\lambda$ are well predicted by linear regression as there is a perfect direct proportionality. 
Was zu erwarten war. 


Do multi linear reg like in Peprah et al. 2017\cite{peprah2017appraisal}.

%MSE(data=all) =427
%MSE(data=emma)=250

%\iffalse
%%%%%%%%%%%%%%%%%%%%%%%%%%%%%%%%%%%%%%%%%%%%%%%%%%%%%%%%%%%%%%%%%%%%%%%%%%%%%%%%%%%%%%%%5
\subsubsection{Grid Search}
For \gls{krr} and {svm} I had to find out what the ideal hyper parameters were. 
So, I anstellen a grid search with the following hyperparameters: 
%kernel=["poly","rbf"]#,"sigmoid"]
%C= np.array(range(1,20))*0.05
%degree=range(1,6)
%epsilon= np.array(range(1,20))*0.2
%gamma=[0.0,0.1,1.0,10.0]
%#param = {"kernel":kernel, "C":C, "degree":degree, "epsilon":epsilon, "gamma":gamma }
%C=[1,0.1]; degree=[3]; epsilon=[1,2]; gamma=['scale']
Now what are C,epsilon and gamma? 

Since the data set was very small, I could be generous with the grid search even on relatively humble hardware (intel i7-8550U). 

\begin{table}[htb]
	\centering
    \caption{Post EMMA vergleich}
	\label{tab:post-emma}
	\begin{tabular}{c cc cc cc cc}
    \hline\hline
    G&  MAEe&   MSEe&   MAEp&   MSEp&   MAEc&   MSEc&   MAEa& MSEa \\
    \hline
        MARS&   10& 171&    28& 1084&   30& 1280&   17& 548\\
        lin Reg&    13& 250&    24& 747&    34& 1327&   17& 454\\
        ML KRR& 15& 415&    26& 1048&   28& 1586&   19& 676\\
        ML SVM& 15  415&    26& 1050&   28& 1588&   19& 677\\
    \hline\hline
	\end{tabular}
    %
	\begin{tabular}{c cc cc cc cc}
    \hline\hline
    p&  MAEe&   MSEe&   MAEp&   MSEp&   MAEc&   MSEc&   MAEa& MSEa \\
    \hline
        MARS&   0.14&   0.03&   0.41&   0.21&   0.39&   0.18&   0.25&   0.11\\
        lin Reg&    0.19&   0.06&   0.30&   0.15&   0.38&   0.13&   0.24&   0.09\\
        ML KRR& 0.30&   0.22&   0.47&   0.38&   0.52&   0.44&   0.34&   0.26\\
        ML SVM& 0.28&   0.12&   0.37&   0.21&   0.42&   0.25&   0.32&   0.16\\
    \hline\hline
	\end{tabular}
\end{table}
    

%%%%%%%%%%%%%%%%%%%%%%%%%%%%%%%%%%%%%%%%%%%%%%%%%%%%%%%%%%%%%%%%%%%%%%%%%%%%%%%%%%%%%%%%5
\subsubsection{KRR}

%%%%%%%%%%%%%%%%%%%%%%%%%%%%%%%%%%%%%%%%%%%%%%%%%%%%%%%%%%%%%%%%%%%%%%%%%%%%%%%%%%%%%%%%5
\subsubsection{SVM}
\td{plot the predicted data for variables which should be excluded (Tcal, Vcal,Conc, layers)}

%%%%%%%%%%%%%%%%%%%%%%%%%%%%%%%%%%%%%%%%%%%%%%%%%%%%%%%%%%%%%%%%%%%%%%%%%%%%%%%%%%%%%%%%5
\subsubsection{PCR?}

%\fi

\subsubsection{Further}
\begin{itemize}
    \item grid search
    \item KRR (start with grid search) (???)
    \item SVM (start with grid search) (???)
    \item \url{https://blog.minitab.com/en/adventures-in-statistics-2/regression-smackdown-stepwise-versus-best-subsets}
    \item \url{https://blog.minitab.com/en/how-to-choose-the-best-regression-model}
\end{itemize}

%%%%%%%%%%%%%%%%%%%%%%%%%%%%%%%%%%%%%%%%%%%%%%%%%%%%%%%%%%%%%%%%%%%%%%%%%%%%%%%%%%%%%%%%5
\subsubsection{Comparison} 
compare all methods. 
How should I compare them? 
What are the measurables. 

