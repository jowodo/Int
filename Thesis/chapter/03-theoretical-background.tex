%\url{https://link.springer.com/content/pdf/10.1186/2228-5326-3-8.pdf}
\todo{Structure: Free writing} 
Ueber das Material, also PV und speziell CIGS (copper indium gallium sulfide) 
ueber den MC prozess berichten, also DB, i-v curven, sputtering, sem, ir, xrd
ueber den it ml stuff und statistisch, also ML algemein, pso im speziellen 
perfekt fuer intro einfach mal erwaehnen was wo in diesem capitel zu finden sein wird. 
Books PV, analytic otto? 
\\
This chapter aims to shine light on the evolution of PV and 
give some introduction on methods used during experimental and \td{analytical} phase of this work. 
\subsection{Photovoltaics}
\iffalse
The grundlage for all pvs is the photovoltaik effect which was entdeckt by Albert Einstein adn for which he got the nobel price. 
The Prinziple is easy: When the energy (E = hv) of the light is \td{large,strong,high}er 
than the binding energy of an electron the electron is ejected with the remaining energy as kinetic energy 
\begin{math}
	E_{kin}=hv - Eb
\end{math}
Different Materials have different binding energies. 
Metals do have in general lower binding energies than covalent bound material and semiconductors do have even lowers E_b. Really? 
That's Silica is in a lot of PVs. 
The next generation of PVs. 
CIGS has in contrast to silicon based PV a direct band gap\td{source and what does that mean?}
duennschicht pv, haben eine effeftivitaet von 7-16\% vs 15-22\% \cite{Mertens2018}
\fi
%%%%%%%%%%%%%%%%%%%%%%%%%%%%%%%%%%%%%%%%%%%%%%%%%%%%%%%%%%%%%%%%%%%%%%%%%%%%%%%%%%%%%%%%%
\subsubsection{Problems of current energy supply}
The world wide energy consumption has more than doubled between 1970 and 2015\cite{BP2017} 
and according to recent studies both fossil\cite{BGR2017} and uranium sources\cite{Uran2006} 
will be exhausted within the next 100 years. 
%This time period must of course taken with a grain of salt as it can happen that even though the resource are being exhausted they can rise from one year to another as new reserves are being discovered and explored. 
Even though this time period is not exact and highly dependent on detection methods, 
this number is rather small and brings us in zugzwang. 

\subsubsection{History of Photovoltaics}
%aenderung von widestand vonhalbleiter selen von bit ing willouhgby smith und ass joseph may fuer pv relevanten innere photoeket
%1876 william adam und richard Day zeigten mit Selenstab mit platin elektroden dass festkörper Lichenergie direkt in elekt energie umwalndeln 
%1905 erklaertle die physiklischen hinterdr'ünde mit seinem lichtquantentheorie einstein 
%die ersten solarzellen wurden in den 1950s in Nieschen anwendungen wie der Raumfahrt mit einem wirkungsgrad unter 10 prozent 
%durch die oelkriese 1973 stieg das interesse an pv und alternativen energiequellen und 1977 das erste salor...? 
The photoelectric effect was first described in 1839 by french scientist Alexandre Edmond Becquerel\cite{becquerel1839memoire}, the father of Henri Becquerel. 
With the discovery of photo conductivity of selenium
%the correlation between the {change of the }resistance and the irradiating light 
by British engineer Willouhgby Smith\cite{Smith1873Selenium} 
%and his assistant Joseph May 
another relevant piece of the photovoltaic jigsaw was discovered. 
In 1876 William Adams and Richard Day\cite{Adams1876Selenium} showed that 
the energy of light can be directly converted into electrical energy by a bar of 
selenium with attached platinum electrodes.
And finally, in 1905 Einstein described the physical background of the photoelectric 
effect with his light quantum theory\cite{einstein1905erzeugung}.
In the 1950s the first solar cells (with efficiencies under 10 percent) were used in niche applications such as space flight. 
Eventually, the interest in photovoltaic and other alternative energy sources 
rose - fuelled by the oil crisis in 1973 - 
and the development of photovoltaic for the consumer market was boosted. \\

\subsection{CIGS}
%%%
%Materials from the chalcopyrite group (tetragonal crystal system) (eg CuInGaSe(CIGS)) can be used as absorber in such thin film cells.
CIGS ($\text{CuIn}_\text{x}\text{Ga}_{\text{(1-x)}}\text{Se}_2$) is of the chalcopyrite group (tetragonal crystal system) and can be used as thin film PV. 
Just like CdTe, GaAs and amorphoous silicon CIGS have much higher absorption coeffiecients 
(lower penetration depth) of visible light than crystlline silicon (see table \ref{tab:cigs:alpha}). 
These thin film PVs not only use less material, but also can be used in flexible applications. 
%%%
%Amorphous silicon and copper indium selenide (CIS) \td{(see next chapter)} have much higher absorption coefficients of visible light than crystalline silicon (see table \ref{tab:cigs:alpha}) and can therefore be much thinner and use less material. 
%Another advantage of thin film cells is that they can be used in flexible applications. 
%which because of their high absorption coefficient of light %and therefore shallow penetration depth of light%need less material and can be used to produce flexible cells \\%of amorph silicon or CIS (see table p84) 


\begin{table}[htb]
	\small
    \center
    \begin{tabular}{cccccc}
        \hline
        \hline
        Material&   Type&    Band Gap [eV]&    Wavelenght [nm]&    sbsortption coef. alpha&    penetration depth [um]\\
        \hline
        c-Si&   indirect&   1.12&   600&    4000/cm&    2.5\\
        c-Si&   indirect&   1.12&   1000&    64/cm&    150\\
        c-Si&   indirect&   1.12&   1100&    3.5/cm&    290\\
        a-Si&   direct&      1.7&    600&    40,000/cm&  0.25\\
        CdTe&   direct&      1.45&    600&    37,000/cm&  0.3\\
		GaAs&   direct&      1.42&    600&    40,000/cm&  0.2\\
        \hline
        \hline
    \end{tabular}
	\caption{data from \cite{mertens2015photovoltaik}}
	\label{tab:cigs:alpha}
\end{table}

%CIGS\\ 
%Materialien aus der Grupper der Chalkopyrite wisen allesamt eine XXX Gitterstruktur auf genauso wie Namengeber der Gruppe, das Chalkopyrit (CuFeS2). 
%Meist wird ein Material mit der Summenformel $CuIn_XGA_{(1-x)}Se_2$ verwendet. 
%Materials from the chalcopyrite group (tetragonal crystal system) (eg CuInGaSe(CIGS)) can be used as absorber in such thin film cells.
%Mostly a Material with the chemical formula  $\text{CuIn}_\text{x}\text{Ga}_{\text{(1-x)}}\text{Se}_2$ (CIGS).
%The awesome thing about CIGS is that the band gap can be varied by between 1eV and 1.7eV by varying the inidum gallium ratio accordingly. 
The band gap of CIGS can be varied by between 1eV and 1.7eV by varying the inidum gallium ratio.
This is a result of the large difference of band gaps of CuInSe2 and CuGaSe2 (see table \ref{tab:cigs}). 
%Durch den hohen Unterschied der Bandabstaedne von CuInSe2 und CuGaSe2 kamnn durch entsprechende Mischverhaeltnisse von In und Ga der Bandabstand eingestellt werden (Tablep 142) Abbildung Bild 5.22 p143) beschreiben und bild für Modul 
%Because of the large difference of band gaps of CuInSe2 and CuGaSe2 the band gap can be configured by mixing ratio of inidum and gallium (see table \ref{tab:cigs}).

{Bild: electrode + n-ZnO, [2.42eV] n-CdS 40nm, [1.15eV] p-CIGS 1.5um, Molybden 0.5 um, glas substrat, borrow book} 
The standard substrate is glass \td{why?}. A flexible substrate is needed for a flexible module. 
Plastics have a very low metling points and metals are conducting, but can be coated with a non conducting material such as \td{ZrO2}.
\begin{table}[htb]
    \center
    \begin{tabular}{cccc}
        \hline\hline
        Empirical Formula&    Name&   Band Gap&    Abbreviation\\
        \hline
        CuInSe2&       copper indium di selenide&  1&  CISe\\
        CuInS2&        copper indium di sulfide&  1.5&  CIS\\
        CuGaSe2&       copper gallium di selenide&  1.7&  CIGSe\\
        CuGaS2&        copper gallium di sulfide&  1.55&  CIGS\\
        \hline\hline
    \end{tabular}
	\caption{band gaps of different chalcopyrites}
	\label{tab:cigs}
\end{table}


%Um ein Modul flexibel zu machen kann statt Glas Stahl als stbstrat verwendet werden. 
%Allerdings wird ein Isolater zwischen Stahsubstrat und CIGS Zelle benötigt


\subsection{Zirconium oxide}
Zirconium oxide \td{(Zr02)} is ceramic with a bandgap of 5-7 eV and dielectric constant of 15-22\cite{Anwar2017}. 
This makes it attractive as an insulator for semiconductor and \td{PV} industry. 
It is monoclinic below 1170 °C, tetragonal between 1170 °C and 2370 °C, and cubic above 2370 °C. \td{[R. Stevens, 1986. Introduction to Zirconia. Magnesium Elektron Publication No 113]}
%Ralph Nielsen "Zirconium and Zirconium Compounds" in Ullmann's Encyclopedia of Industrial Chemistry, 2005, Wiley-VCH, Weinheim. doi:10.1002/14356007.a28_543
what is important properites of zro2? 

\subsection{Sputtering}
% https://en.wikipedia.org/wiki/Sputter_deposition
\td{what is the principle behind sputtering
what is sputtering, how does it work, what different kinds of sputtering is there, 
what is is used for? 
what was it used for in this work? 
}
%Sputtering is a physical deposition method. 
Sputtering describes the process of particles being sputtered from a surface after 
impact of highly energetic particles.
These high energy particles start cascades of momentum transfers. 
If a path of a cascade reaches back to the surface and the kinetic energy of a surface 
particle exceeds the binding energy, it is ejected (or sputtered).

%Sputtering describes the process when highly energetic particles collide with a surface 
%and through a cascade of momentum transfers surface particles are depleted and sputter from the surface.
This process occurs naturally in outer space (through alpha particles, electrons etc.) 
and is used in science and technology for 
sputter cleaning, thin film deposition, analysis or etching. 
The phenomenon was discovered and identified as an atomic collision phenomenon in the 
middle of the 20th century and now is a field of research with its own niche subfields 
(e.g. sputtering of exotic materials (such as condensed gases, planetary
atmospheres, lunar dust grains, proteins and viruses), high density cascades and 
linear cascades, which could be called classical sputtering)
%sputtering of insulators and biomolecular material)
\cite{Sigmund1987}.
%
In this work it was used to deposit thin gold contacts through a mask. 
A directional current sputtering machine was used. 
\td{describe the machine}
A argon plasma in a strong electrical field (some kV) provides the highly energetic particles. 
Argon ions are accelerated through the field to the source (a metal wafer), which acts as cathode, and start cascades. 
In order to maximise the yield (sputtering speed) a magnet is used to trap the plasma close to source. 
Plasma sputtering with a strong magnet to 
1.5h 
\\
\td{A high voltage is angelegt at the source (a metal wafer). 
Electrons with a high kinetic energy bump into the source and herausschießen metal atoms, 
which will deposit on a target. 
By ausrüsten the target with a mask, arbitrary musters can be obtained (e.g. contact arrays).
}
How are the electrons erzeugt? 
\td{With a magnetron (a micro wave erzeuging high vacuum bauteil) a plasma is sustained. 
}
Find source

\subsection{SEM}
\url{https://doi.org/10.1016/B978-0-12-816806-6.00017-0}\\
what is sputtering, how does it work, what different kinds of sputtering is there, what is is used for? 
\subsection{Infrared absorption}
\td{IR around page 45;
REFLECTIVITY? can also be used for determination of thickness p47
}
\subsection{X-Ray Diffraction}
\url{https://doi.org/10.1016/B978-0-12-816806-6.00017-0}\\
\url{https://chem.libretexts.org/Courses/Franklin_and_Marshall_College/Introduction_to_Materials_Characterization__CHM_412_Collaborative_Text/Diffraction_Techniques/X-ray_diffraction_(XRD)_basics_and_application}\\
\subsection{Particle Swarm Optimization}
\subsection{Machine Learning}
\subsubsection{History}
\subsubsection{different kinds} 
supervised vs unsupervised
classification (discrete) vs regression
\subsection{Princlipal Component Analysis}
reinforcement learning could also have been a nice option
\subsection{Linear Regression}
