%\url{https://link.springer.com/content/pdf/10.1186/2228-5326-3-8.pdf}
\todo{Structure: Free writing} 
Ueber das Material, also PV und speziell CIGS (copper indium gallium sulfide) 
ueber den MC prozess berichten, also DB, i-v curven, sputtering, sem, ir, xrd
ueber den it ml stuff und statistisch, also ML algemein, pso im speziellen 
perfekt fuer intro einfach mal erwaehnen was wo in diesem capitel zu finden sein wird. 
Books PV, analytic otto? 
\subsection{Photovoltaik and CSGI}
The grundlage for all pvs is the photovoltaik effect which was entdeckt by Albert Einstein adn for which he got the nobel price. 
The Prinziple is easy: When the energy (E = hv) of the light is \td{large,strong,big,high}er than the binding energy of an electron the electron is ejected with the remaining energy as kinetic energy 
\begin{math}
	E_{kin}=hv - Eb
\end{math}
Different Materials have different binding energies. Metalls do have in general lower binding energies than covalent bound material and halbleiter do have even lowers E_b. 
Thats Silica is in a lot of PVs. 
The next generation of PVs. 
CIGS has in contrast to silicium based PV a direct band gap\td{source and what does that mean?}
duennschicht pv, haben eine effeftivitaet von 7-16\% vs 15-22\% \cite{Mertens2018}
\subsection{Zirconium oxide}
\subsection{Sputtering}
%https://en.wikipedia.org/wiki/Sputter_deposition
\subsection{SEM}
\url{https://doi.org/10.1016/B978-0-12-816806-6.00017-0}\\
\subsection{Infrared absorption}
\td{IR around page 45;
REFLECTIVITY? can also be used for determination of thickness p47
}
\subsection{X-Ray Diffraction}
\url{https://doi.org/10.1016/B978-0-12-816806-6.00017-0}\\
\url{https://chem.libretexts.org/Courses/Franklin_and_Marshall_College/Introduction_to_Materials_Characterization__CHM_412_Collaborative_Text/Diffraction_Techniques/X-ray_diffraction_(XRD)_basics_and_application}\\
\subsection{Particle Swarm Optimization}
\subsection{Machine Learning}
\subsubsection{History}
\subsubsection{different kinds} 
supervised vs unsupervised
classification (discrete) vs regression
\subsection{Princlipal Component Analysis}
reinforcement learning could also have been a nice option
