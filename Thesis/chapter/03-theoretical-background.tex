%%%%%%%%%%%%%%%%%%%%%%%%%%%%%%%%%%%%%%%%%%%%%%%%%%%%%%%%%%%%%%%%%%%%%%%%%%%%%%%%%%%%%%%%%
%%%%%%%%%%%%%%%%%%%%%%%%%%%%%%%%%%%%%%%%%%%%%%%%%%%%%%%%%%%%%%%%%%%%%%%%%%%%%%%%%%%%%%%%%
%%%%%%%%%%%%%%%%%%%%%%%%%%%%%%%%%%%%%%%%%%%%%%%%%%%%%%%%%%%%%%%%%%%%%%%%%%%%%%%%%%%%%%%%%

%\td{https://en.wikipedia.org/wiki/Thesis}
\ds{what is it? What is it used for? How does it work? What different kinds are there? }
\ds{what? how? why?}
This chapter can be broken down into three sections. 
The first section tries to shine light on the evolution of PV and give some background on \gls{cigs}.
The middle section reads about material-scientific methods which were used during the practical part of this work. 
The last and third part focuses on the information technological, algorithmic and analytical methods used to optimize and predict material properties. 

%%%%%%%%%%%%%%%%%%%%%%%%%%%%%%%%%%%%%%%%%%%%%%%%%%%%%%%%%%%%%%%%%%%%%%%%%%%%%%%%%%%%%%%%%
%%%%%%%%%%%%%%%%%%%%%%%%%%%%%%%%%%%%%%%%%%%%%%%%%%%%%%%%%%%%%%%%%%%%%%%%%%%%%%%%%%%%%%%%%%%%%
%%%%%%%%%%%%%%%%%%%%%%%%%%%%%%%%%%%%%%%%%%%%%%%%%%%%%%%%%%%%%%%%%%%%%%%%%%%%%%%%%%%%%%%%%
\subsection{Photovoltaics}
%
\subsubsection{Problems of current energy supply}
The world wide energy consumption has more than doubled between 1970 and 2015\cite{BP2017} 
and according to recent studies both fossil\cite{BGR2017} and uranium sources\cite{Uran2006} 
will be exhausted within the next 100 years. 
Even though this time period is not exact and highly dependent on detection methods of resources, 
this number is rather small and brings us in zugzwang to develop sustainable energy sources. 
One viable option is \gls{pv}.

%%%%%%%%%%%%%%%%%%%%%%%%%%%%%%%%%%%%%%%%%%%%%%%%%%%%%%%%%%%%%%%%%%%%%%%%%%%%%%%%%%%%%%%%%%%%%
\subsubsection{History of Photovoltaics}
The photoelectric effect was first described in 1839 by french scientist Alexandre 
Edmond Becquerel\cite{becquerel1839memoire}, the father of Henri Becquerel (the person after whom the unit is named).
Another relevant piece of the \gls{pv} jigsaw was brought to light
%discovered
with the discovery of photo conductivity of selenium
by British engineer Willouhgby Smith\cite{Smith1873Selenium}.
In 1876 William Adams and Richard Day\cite{Adams1876Selenium} showed that 
the energy of light can be directly converted into electrical energy by a bar of 
selenium with attached platinum electrodes.
And finally, in 1905 Einstein described the physical background of the photoelectric 
effect with his light quantum theory\cite{einstein1905erzeugung}.
In the 1950s the first solar cells (with efficiencies under 10 percent) were used in niche applications such as space flight. 
Eventually, the interest in photovoltaic and other alternative energy sources 
rose - fuelled by the oil crisis in 1973 - 
and the development of photovoltaic for the consumer market was boosted. 
This development lead to a drop in 
average price for \gls{pv} module from \$ 100 per watt in 1975 to \$ 4 per watt in 2007\cite{pagliaro2008flexible}.

%%%%%%%%%%%%%%%%%%%%%%%%%%%%%%%%%%%%%%%%%%%%%%%%%%%%%%%%%%%%%%%%%%%%%%%%%%%%%%%%%%%%%%%%%%%%%
\subsubsection{Photovoltaics Basics}
The process of conversion of photons into voltaic energy can be broken down into two essential steps (both in silicon based \gls{pv} or in plants): 
the creation of an electron-hole pair and then its separation by the structure of the material.\cite{markvart2013principles}
This means that \gls{pv} cells are basically diodes, which have a low resistance in one direction and a high resistance in the other direction. Diodes are also 
%This is often achieved through a p-n junction.\cite{breitenstein2013understanding}
This is often achieved through a p-n junction.
The n-type side has excess electrons and p-type side has excess electron holes. 
If the n- and p-type are of the same basis material, they are called homojunctions (silicon); if not, heterojunctions e.g. \ch{CdS}/\gls{cigs} and \ch{CdS/CdTe}.\cite{breitenstein2013understanding}
%
The first marketable \gls{pv} were crystalline silicon photovoltaic modules which still have the biggest market share in the \gls{pv} segment including polycrystalline and monocrystalline silicon.
Together with amorphous silicon 2010 over 88\% of all sold \gls{pv} where made of silicon\cite{breitenstein2013understanding}.
A comprehensive overview over different \gls{pv} technologies can be found in \cite{markvart2013principles}.


%%%%%%%%%%%%%%%%%%%%%%%%%%%%%%%%%%%%%%%%%%%%%%%%%%%%%%%%%%%%%%%%%%%%%%%%%%%%%%%%%%%%%%%%%%%%%
\subsubsection{CIGS}
\gls{cigs} (\ch{CuIn_{x}Ga_{1-x}Se2}) is of the chalcopyrite (\ch{CuFeS2}) group (tetragonal crystal system). 
%It was 
\ch{CuInSe2} and \ch{CuGaSe2} were first synthesised in 1953 by Harry Hahn et al.\cite{hahn1953untersuchungen}.
The potential use of \ch{CuInSe2} as \gls{pv} material in combination with \ch{CdS} was first mentioned in 1974\cite{wagner1974cuinse2}.
%Already in 1975 efficiencies of cells of over 10\% were achieved.\cite{kazmerski1976thin}
Cells with efficiencies of over 10\% were achieved already by 1975.\cite{kazmerski1976thin}
Today, 
\gls{cigs} modules (multiple \gls{pv} cells in series) 
reach efficiencies of up to 16\%\cite{feurer2017cigs},
monocrystalline silicon \gls{pv} modules reach efficiencies of 14-20\%
and polycrystalline silicon \gls{pv} modules 12-17\%\cite{mcevoy2011practical}.

Just like CdTe, GaAs and amorphous silicon CIGS has much higher absorption coefficients 
(lower penetration depth) of visible light than crystalline silicon (see table \ref{tab:cigs:alpha}). 
This is due to a direct band gap rather than an indirect band gap (like crystalline silicon). 
These thin film \gls{pv}s not only use less material, but also can be used in flexible applications. 

\begin{table}[tbh]
	\small
    \center
    \begin{tabular}{cccccc}
        \hline
        \hline
		Material&   Type&    Band Gap [\ev{}]&    Wavelength [\nm{}]&    Absorption coef. $\alpha$ [\pcm{}]    &Penetration Depth [\um{}]\\
        \hline
		c-Si&   indirect&   1.12&   600&    \num{4000}&    2.5\\
		c-Si&   indirect&   1.12&   1000&    \num{64}&    150\\
		c-Si&   indirect&   1.12&   1100&    \num{3.5}&    290\\
		a-Si&   direct&      1.7&    600&    \num{40000}&  0.25\\
		CdTe&   direct&      1.45&    600&    \num{37000}&  0.3\\
		GaAs&   direct&      1.42&    600&    \num{40000}&  0.2\\
        \hline
        \hline
    \end{tabular}
	\caption{data from \cite{mertens2015photovoltaik}}
	\label{tab:cigs:alpha}
%
	\vspace{1cm}
    \begin{tabular}{cccc}
        \hline\hline
		Empirical Formula&    Name&   Band Gap [\si{eV}{}]&    Abbreviation\\
        \hline
		\ch{CuInSe2}&       copper indium di selenide&  1.04&  CISe\\
		\ch{CuInS2}&        copper indium di sulfide&  1.5&  CIS\\
		\ch{CuGaSe2}&       copper gallium di selenide&  1.7&  CIGSe\\
		\ch{CuGaS2}&        copper gallium di sulfide&  1.55&  CIGS\\
        \hline\hline
    \end{tabular}
	\caption{band gaps of different chalcopyrites}
	\label{tab:cigs}
\end{table}

\begin{figure}
	\includegraphics[width=\textwidth]{./Pics/cigs.png}
	\caption{Schematic layer sequence of a standard \ch{ZnO}/\ch{CdS}/\ch{Cu(In,Ga)Se2} thin-film solar cell.}
	\label{fig:cigs}
\end{figure}

%\td{Bild: electrode + n-ZnO, [2.42eV] n-CdS 40nm, [1.15eV] p-CIGS 1.5um, Molybden 0.5 um, glass substrate, borrow book (Recreated based on\cite{mertens2015photovoltaik}) or (Image source:\cite{mertens2015photovoltaik})}
%\td{On the top of a \gls{cigs} cell there is a conducting and durchsichtige layer of \ch{ZnO}. 
%Why is there this doped \ch{ZnO} layer? 

Amazingly, the band gap of CIGS can be varied between \ev{1} and \ev{1.7} by varying the indium gallium ratio.
This is a result of the large difference of band gaps of \ch{CuInSe2} and \ch{CuGaSe2} (see table \ref{tab:cigs}). 
%%
%%%%%%%%%%%%%%%%%%%%%%%%%%%%%%%%%%%%%%%%%%%%%%%%%%%%%%%%%%%%%%%%%%%%%%%%%%%%%%%%%%%%%%%%%%%%%
In figure \ref{fig:cigs} the schematic layer sequence of a standard \gls{cigs} thin film cell is shown. 
Typically a \um{1} thick molybdenum layer is deposited on soda lime glass. 
The sodium in the glass diffuses through the molybdenum layer and increases efficiency and reliability by directing the growth of \gls{cigs} in 112 direction\cite{hedstrom1993cigs}.
A \gls{cigs} layer of 1-\um{2} thickness is applied on top via co-evaporated from elemental sources controlled by a mass spectrometer.\cite{hedstrom1993cigs}
The p-type doping of \gls{cigs} is achieved by adding more than stoichiometric copper to the mix. 
The heterojunction is then completed by deposition of \ch{CdS} (typically \nm{50} thick).
\ch{CdS} was earlier used as front contact, but now only acts as n-type wide-gap window and buffer. 
\ch{ZnO} has a band gap of \ev{3.2} and is therefore transparent for visible light. 
The \ch{ZnO} window layer (usually of thickness 50-\nm{70} is highly conductive (especially the \ch{Al} doped layer) and acts as front contact. 

%
Instead of glass as substrate, steel and plastics (e.g. polyimide\cite{feurer2017cigs}) can be used to create flexible \gls{cigs} modules. 
They both come with their own inconveniences. 
Plastics generally have very low melting points which restrict preparation temperatures.
Steel, on the other hand, is temperature resistant enough, but is an electric conductor. 
The conductance of steel can be masked by a insulator layer (e.g. \ch{ZrO2}, \ch{Al2O3}).

%%%%%%%%%%%%%%%%%%%%%%%%%%%%%%%%%%%%%%%%%%%%%%%%%%%%%%%%%%%%%%%%%%%%%%%%%%%%%%%%%%%%%%%%%%%%%
%%%%%%%%%%%%%%%%%%%%%%%%%%%%%%%%%%%%%%%%%%%%%%%%%%%%%%%%%%%%%%%%%%%%%%%%%%%%%%%%%%%%%%%%%%%%%
%%%%%%%%%%%%%%%%%%%%%%%%%%%%%%%%%%%%%%%%%%%%%%%%%%%%%%%%%%%%%%%%%%%%%%%%%%%%%%%%%%%%%%%%%%%%%
\subsection{Materials and Their Analysis}
%%%%%%%%%%%%%%%%%%%%%%%%%%%%%%%%%%%%%%%%%%%%%%%%%%%%%%%%%%%%%%%%%%%%%%%%%%%%%%%%%%%%%%%%%
\subsubsection{Zirconium oxide}
Zirconium oxide \gls{zro} is a ceramic with a band gap of 5-7 eV and dielectric constant of 15-22 at room temperature\cite{Anwar2017}. 
This makes it attractive as an insulator for semiconductor and \gls{pv} industry. 
It is monoclinic below 1050 °C, tetragonal between 1170 °C and 2370 °C, and cubic above 2370 °C\cite{Nielsen2005}.
The cubic phase can be stabilized down to room temperature by the addition of magnesia (\ch{MgO}), calcia (\ch{CaO}) or yttria (\ch{Y2O3}) which avoids mechanical failing due to shrinkage 
when cooling and undergoing phase transition\cite{Nielsen2005}.
It is very resistant to acids (except \ch{HF} and hot \gls{h2so4}) and alkalis\cite{Nielsen2005}.
Hydrous Zirconium Oxide (\ch{Zr(OH)8 * 16 H2O}) gel can be produced by neutral hydrolysis of sodium zirconate (\ch{NaZrO3}). 
"Zirconium alkoxides hydrolyze quite easily, [providing] a route to high purity, high-surface-area zirconium oxide"\cite{Nielsen2005}.

%%%%%%%%%%%%%%%%%%%%%%%%%%%%%%%%%%%%%%%%%%%%%%%%%%%%%%%%%%%%%%%%%%%%%%%%%%%%%%%%%%%%%%%%%%%
\subsubsection{Sol-Gel and Doctor Blading}
%\td{what is it? What is it used for? How does it work? What different kinds are there? }One of the advantages of sol-gel process is that it can be used in roll-to-roll procedures.
%
\Gls{dbc} Doctor blade coating (also known as tape casting) is widely used in the textile, paper, photographic film, printing, and ceramic industries.
The roll to roll compatible process can fabricate highly uniform flat films over large areas\cite{yang2010large}.
A blade is moved over a substrate spreading a slurry at a fixed distance with a fixed velocity.
In roll-to-roll processes the substrate moves instead of the blade. 

%%%%%%%%%%%%%%%%%%%%%%%%%%%%%%%%%%%%%%%%%%%%%%%%%%%%%%%%%%%%%%%%%%%%%%%%%%%%%%%%%%%%%%%%%%%
\subsubsection{Sputtering}
%SPUTTERING FREEWRITING: 
%what is sputtering? \\
Sputtering is the processes of highly energetic ions hitting a surface and atoms or molecules being expelled. 
This is called a physical vapor deposition (PVD) technique. 
PVD can be divided into activation by thermal energy and activation by energetic particle bombardment. 
Sputtering is of the latter, which 
is advantageous if substrates can't withstand high temperatures.
%\td{what can it be used for?}\\
Sputtering evolved from being a curious experiment in the middle of the 20th century to having various applications in research and engineering.
Use cases vary from thin films depositions for \gls{pv}, for electrical circuits or for storage media such as CDs and DVDs 
over sputter cleaning and etching to analysis.
Advantages of 
sputtered thin films include good adhesion to the substrate and good step coverage\cite{Swann1988}.

\begin{figure}[htb]
	\includegraphics[width=\textwidth]{./Pics/sputter0.png}
    \caption{Propagation of energy in a dense cascade. Left graph: "Primary" shockwave. Right graph: "Thermal" pulse or "secondary" shockwave. Figure taken from Sigmund 1987\cite{sigmund1987}}
	\label{fig:sputter}
\end{figure}

%\td{how does it work?}\\
A high voltage is applied to 
two parallel electrodes with low pressure gas in between. 
The target acts as cathode and the substrate (holder) as anode (see figure \ref{fig:sputter}).
%the target which acts as a cathode with the substrate as anode in a parallel geometry (see \td{fig}).
The cathode, then, emits electrons which collide with a gas particles (mostly argon because of it's inert properties and potential to transfer more kinetic than lighter noble gases). 
Some gas particles may get ionized by the collision and the gas cations are accelerated to the cathode. 
If a cation has enough energy it will bump off one or more atoms or molecules from the surface. 
This happens by a cascade of momentum transfers, which can reach the surface again (see fig \td{from \cite{sigmund1987}}). 
If a surface particle obtains momentum pointing away from the bulk and its kinetic energy is higher than the binding energy, the particle is sputtered. 
This sputtered particle travels unaffected by the electrical field perpendicular to the surface towards the substrate and condenses with other particles to form a layer.
The pressure should be small, such that the sputtered particle has a long \gls{mfp}, but on the other hand 
a minimum pressure is needed to keep the plasma "alive". 
Usual pressures are around \SI{1}{\Pa} (\num{e-2}\SI{}{\milli\bar}) or lower\cite{Swann1988}.

A magnetron can be placed behind the cathode (target) in order to trap ejected electrons close to the source. 
This prevents high energy electrons from reaching the target and undoing the deposited layer and this also increases the probability of an electron colliding with an argon atom and ionizing it, increasing the yield.

When oxygen or nitrogen are added to the  gas this is called reactive sputtering.
Sputtered atoms will react with the gas and result in oxide or nitrides layers, respectively.
The stoichiometry of the resulting layer can be regulated by gas ratios, but too much reactive gas can lead to target poisoning. 
Meaning the target begin covered by an insulating layers which can lead to defects in the growing film\cite{Kelly2000}. 

The limitation of only being able to use conducting materials as targets can be circumvented by using a radio frequency electrical field. 
This prevents a charge building up on the target. %\td{why is a charge?}
Although RF sputtering is more versatile, DC sputtering is more common because of it's simpler system and economical reasons.
%\td{sputtered particles are neutral and not influenced by the electrical field}

%%%%%%%%%%%%%%%%%%%%%%%%%%%%%%%%%%%%%%%%%%%%%%%%%%%%%%%%%%%%%%%%%%%%%%%%%%%%%%%%%%%%%%%%%
\subsubsection{Scanning electron Microscopy}
The history of \gls{sem} can be traced back to 1843 when Scottish clockmaker Alexander Bain filed a patent for dissecting an image by scanning. A detailed history of \gls{sem} can be read in a open-to-read paper by McMullan from 1995\cite{McMullan1995}. 
\Gls{sem} is a microscopical technique which allows visualisation of surfaces with features in the nano meter regime. 
While optical microscopes use visible light and optical lenses, \gls{sem} uses accelerated electron beams and electrostatic and electromagnetic lenses.
This allows the generation of much more detailed images due to the shorter wavelengths of electrons compared to light\cite{Kaliva2020}.
The electron beam produces X-rays, elastically backscattered (primary) electrons, inelastic (secondary) electrons and Auger electrons. 
Secondary electrons carry information to conclude morphology and topology of the sample while X-rays can be used to identify the elements. 
Electrons originate from either \gls{feg}, where are strong electrical field rips electrons from the bulk, or from thermionic guns where the filament (tungsten W or \ch{LaB6} (brighter and longer lasting but more expensive)) is heated until electrons are emitted. 
Electrons are then accelerated by a voltage of 2 to 40 kV and bundled into narrow beams\cite{Vernon2000} by lenses.
A high \gls{mfp} is needed for electrons to travel from the source to the sample (and to the detector). 
Thus, a very low pressure is in the inside. 
In this work \gls{sem} was used as a preliminary way of checking the surface condition. 


%%%%%%%%%%%%%%%%%%%%%%%%%%%%%%%%%%%%%%%%%%%%%%%%%%%%%%%%%%%%%%%%%%%%%%%%%%%%%%%%%%%%%%%%%
\subsubsection{Infrared Absorption and Spectroscopy}
%%% WHAT? 
\Gls{ir} spectroscopy is a molecular spectroscopic method using interactions of \gls{ir} light (wavelengths $\lambda$ of \num{e-3} to \num{e-6}\m{}
or wave numbers $\bar{\nu}$ of 500 to \pcm{4000} ) with molecules. \cite{Schwedt2008}
In general, light is described as periodic \gls{em} wave 
which - in vacuum - moves with the speed of light ($c = \mps{299792458}$).
The relation of energy~$E$ of a photon, its frequency~$\nu$, wavelength~$\lambda$ and wave number~$\bar{\nu}$ are as follows:
\begin{align*}
	E &= h \cdot \nu \\
	\nu &= c/ \lambda \\
	\bar{\nu} &= 1/\lambda,
\end{align*}
where $h$ is the Planck's constant.
%In words: shorter wavelength (higher frequency) photons are more energetic.
In practice the spectrum of \gls{em} waves is sectioned into different ranges (\td{see figure?+table});
from high to low energy: X-ray, \gls{uv}, \gls{vis}, (near, middle and far) \gls{ir}, microwaves and radio waves. 
X-rays interact with core electrons, \gls{uv}\gls{vis} with valence electrons, \gls{ir} with molecular vibrations, microwaves with molecular rotations and radio with electron and nuclear spins. 
%Molecular vibrations can be separated in valence and deformation vibrations.
%
A molecule has $F=3N$ ($N$ number of atoms) degrees of freedom, including translational $F_T=3$ and rotational $F_R=3$ (2 for linear molecules) movements. 
%
The number of vibrations can therefore calculated as 
\begin{align*}
	F_V &= 3N - F_T - F_R = 3N - 6 \\
	F_V &= 3N - F_T - F_R = 3N - 5 \textrm{ (for linear molecules)}.
\end{align*}
Vibrations are classified in valence vibrations (change of bond length) and deformation (change of bond angle) vibrations\cite{Melker2006}. 
Only vibrations that change the dipole moment of the molecule are \gls{ir} active. 

%\td{describe simple IR with monochromator}

\paragraph{(Fourier Transform) Infrared spectrometer}
 A ceramic material (Nernst glower) is heated to around \oc{1600} as light source. 
 In the two-beam-spectrometer the light is split, sent through the sample and a reference and one of the two beams is alternately sent through a monochromator to a detector (often a thermopile).
%
In the \gls{ft}\gls{ir} spectrometer the beam is sent through the sample, split and 
reflected from a static and from a moving mirror, recombined and detected by a photo 
multiplier (a device which transforms photons into electrical signals). 
How the two beams will interfere upon recombination depends on the optical path difference (also called retardation) of the two light beams\cite{Schwedt2008}.
In a \gls{ft}\gls{ir} spectrometer the reference has to be measured before the sample.
%
When the refractive index of two layers differ, \gls{ir} can be used to measure the thickness\cite{Dumin1967} of optical less dense material.

%\td{Frank-Condon rules, spin verbot and symmetrie verbot are ausschlaggebend für the resulting spectrum. }
%differences to \gls{uv}\gls{vis}: transmission instead of absorbance, higher energy left instead right in plot, base line on top instead of bottom.

\td{
	IR around page 45;
The absorbance intensity is dependent on wave length and molecule structure. 
Absorbance per din 1349: $A(\lambda) = \log{\frac{\phi_{in}}{\phi_{ex}}}$ and transmission $\tau =T=D= \phi_{ex}/\phi_{in}$
}


%%%%%%%%%%%%%%%%%%%%%%%%%%%%%%%%%%%%%%%%%%%%%%%%%%%%%%%%%%%%%%%%%%%%%%%%%%%%%%%%%%%%%%%%%
\subsubsection{X-Ray Diffraction}
%\url{https://link.springer.com/content/pdf/10.1186/2228-5326-3-8.pdf}
\gls{xrd} is used to study the crystalline structure of materials.
Since X-rays wavelengths (\num{0.2} to \nm{10}) are comparable to the interatomic spacing of crystalline solids the beams get reflected and contain information about the structure\cite{Kaliva2020}.
Each crystalline material has a discreet atomic structure, which upon irradiation with 
X-rays causes constructive and destructive interference according to Bragg's law and generates unique diffraction patterns. 
\Gls{xrd} diffraction plots of crystalline materials feature distinct peaks, whereas amorphous materials exhibit a broad curve with a maximum extending over several degrees (2$\theta$).

%\url{https://chem.libretexts.org/Courses/Franklin_and_Marshall_College/Introduction_to_Materials_Characterization__CHM_412_Collaborative_Text/Diffraction_Techniques/X-ray_diffraction_(XRD)_basics_and_application}\\


%%%%%%%%%%%%%%%%%%%%%%%%%%%%%%%%%%%%%%%%%%%%%%%%%%%%%%%%%%%%%%%%%%%%%%%%%%%%%%%%%%%%%%%%%
%%%%%%%%%%%%%%%%%%%%%%%%%%%%%%%%%%%%%%%%%%%%%%%%%%%%%%%%%%%%%%%%%%%%%%%%%%%%%%%%%%%%%%%%%
%%%%%%%%%%%%%%%%%%%%%%%%%%%%%%%%%%%%%%%%%%%%%%%%%%%%%%%%%%%%%%%%%%%%%%%%%%%%%%%%%%%%%%%%%
\subsection{Machine Learning and Statistics}
\ds{
statistics:
	what is the difference between ml and stat? 
		superficially very similar but different goals and diff rahmenbedingungen 
    how can i use ANOVA 
    linear regression 
}

%%%%%%%%%%%%%%%%%%%%%%%%%%%%%%%%%%%%%%%%%%%%%%%%%%%%%%%%%%%%%%%%%%%%%%%%%%%%%%%%%%%%%%%%%
%Why introduce machine learning to this project? 
%I wanted to apply and delve into what I've been studying during my undergrad courses. 
%This seemed like the perfect oppurtunity. 
%%%%%%%%%%%%%%%%%%%%%%%%%%%%%%%%%%%%%%%%%%%%%%%%%%%%%%%%%%%%%%%%%%%%%%%%%%%%%%%%%%%%%%%%%
%Is machine learning just hyped statistics? No! 
		%\href
		%{https://doi.org/10.1177\%2F1352458520978648}
		%{Machine and deep learning in MS research are just powerful statistics – No}
Although machine learning uses several statistical methods, their goal and frame conditions are different. 
%\td{The goal of machine learning is to ???}

%\begin{itemize}
%	\item \td{let GPT-3 write}
%	\item check out links at \url{https://yewtu.be/watch?v=PqbB07n\_uQ4}
%	\item see pic from \url{https://towardsdatascience.com/notes-on-artificial-intelligence-ai-machine-learning-ml-and-deep-learning-dl-for-56e51a2071c2}
%	\item 
%		\href
%		{https://doi.org/10.1177\%2F1352458520978648}
%		{Machine and deep learning in MS research are just powerful statistics – No}
%\end{itemize}

%\td{
%\url{https://stats.stackexchange.com/questions/5026/what-is-the-difference-between-data-mining-statistics-machine-learning-and-ai}
%In principle both \gls{ml} and statistics use math to get information out of data. 
%AI might use machine learning to create "intelligent" acting (playing a game or driving a  car). 
\Gls{ml} tries to predict unseen data points and statistics is a subfield of maths 
which tries to get insight into a given data. 
For example, in a statistical model, it is desirable to reduce the number of inputs. 
This allows the statistician to better study how a change in the input variable can be 
directly affected by the output variable.\cite{gontcharov2019}
%pic from \url{https://towardsdatascience.com/notes-on-artificial-intelligence-ai-machine-learning-ml-and-deep-learning-dl-for-56e51a2071c2}
%}
%What is statistics, though, and what seperates it from \gls{ml}? 
More precisely, mathematical statistics (stochastic)\cite{haertler2014statistisch} is the subject of of 
\td{finding correlation from data, finding mathematical models to describe the data} 
where as (classical) statistics is the domain of representing the data. 
\td{talk about statistics, then the difference to Ml} 
%%%%%%%%%%%%%%%%%%%%%%%%%%%%%%%%%%%%%%%%%%%%%%%%%%%%%%%%%%%%%%%%%%%%%%%%%%%%%%%%%%%%%%%%%%%%%
%\td{if i want to learn about support vector machine, read chapter 4,7,8 from "Learning from Data" by Cherkassky\cite{cherkassky1998learning} (TUBib 726-1998)}
%%%%%%%%%%%%%%%%%%%%%%%%%%%%%%%%%%%%%%%%%%%%%%%%%%%%%%%%%%%%%%%%%%%%%%%%%%%%%%%%%%%%%%%%%%%%%

%%%%%%%%%%%%%%%%%%%%%%%%%%%%%%%%%%%%%%%%%%%%%%%%%%%%%%%%%%%%%%%%%%%%%%%%%%%%%%%%%%%%%%%%%
\subsubsection{Artificial Inteligence and Machine Learning}
%How can \gls{ai} and \gls{ml} be defined? 
\Gls{ai} is a trans-disciplinary field with roots in logic, statistics, cognitive psychology, decision theory, neuroscience, linguistics, cybernetics, and computer engineering.\cite{howard2019artificial}
The history of \gls{ai} goes back to the middle of the \nth{20} century. 
Researchers from the emerging field came together at the Dartmouth conference and the term "\gls{ai}" was coined\cite{McCarthy1955}. 
\Gls{ai}'s  history is beautifully depicted by McCorducks' 1982 book "Machines Who Think"\cite{McCorduck1982,Apter1982}, which focuses on the great minds behind the advances.
Pioneers like Alan Turing thought a lot about how to define, test and implement \gls{ai}\cite{howard2019artificial}. 
One example how to measure \gls{ai} is to let it play chess against a human\cite{Silver2017} 
(in 1997 a chess computer called Deep Blue won against the World Chess Champion Garry Kasparov for the first time\cite{Feng1999}).
%Another test ingenioused by Alan Turing, is that a human communicates with an unknown entity (in written form) and must judge if they are dealing with a machine or a human being. 
Another test ingenioused by Alan Turing, is the imitation game\cite{turing1950imitation}, now a days known as Turing test.
An interrogator communicates with two unknown entities \textbf{A} and \textbf{B} (a woman and a man) and must find out who is who. 
\textbf{A} will try to make to interrogator misjudge, whereas \textbf{B} is on the interrogators side.
The question is if \textbf{A} is replaced with a computer how the ratio of outcomes would deviate from the original ratio. 
%
%(in written form) and must judge who is the 
%if they are dealing with a machine or a human being. 
%
%
%\ds{It is easy for humans to come up with question to detect an AI, 
%but when reading AI written articles\cite{gpt2020}, 
%it's easy to see this test being passed in the near future.}
%
At the moment it's hard to imagine a computer getting a higher ratio than a human
but when reading AI written articles\cite{gpt2020}, it's easy to see this test being passed in the future.

But fear not, that doesn't mean that computers are sentient or more intelligent than humans\cite{searle1980,searle1999married} and certainly not that research is over. 
AI is still a young field, which is strongly growing and is gaining ubiquitous status. 
It is slowly creeping into every aspect of modern human life just like electricity around one hundred years ago. 
%Now, we can't - it's hard to - imagine to live without electricity. 
Realms in which AI is gaining traction are: 
%
playing board games (and beating humans)\cite{Silver2017,Feng1999,Campbell2002}, 
image recognition (very popular for medical diagnosis)\cite{Li2020,Deo2015,Topol2019,Fujiyoshi2019}, 
chemistry\cite{Westermayr2019,goh2017chemception,jha2018elemnet}, 
cyber security\cite{Sarker2021},
facial recognition (to prevent theft of toilet paper \cite{Andrews2017}),
financial sector (as robo-advisors)\cite{Littman2021},
natural language processing (NLP)\cite{Koroteev2021,Liu2021gpt,Parviainen2021} 
(which can also create code and pictures through scalable vector graphics (SVG))
and even creative tasks like 
creating non existing faces\cite{Mansourifar2020}, 
create graphic artwork (DALL-E 2)\cite{Marcus2022} or 
making video games\cite{Guzdial2016}.
%
It is nearly hard to find a field where \gls{ai} isn't used in some way. 
This steady incorporation of \gls{ai} leads to the so called \gls{ai} effect\cite{McCorduck1982,ai100}: 
certain fields get incorporated into \gls{ai} research and practice.
Such that, after some time of general use it is no more considered \gls{ai} (e.g. spam filter or web searches).
Google CEO Sundar Pichai even goes as far and said: 
"AI is one of the most important things humanity is working on. It is more profound than [...] electricity or fire"\cite{Hassan2020}

%
%%%%%%%%%%%%%%%%%%%%%%%%%%%%%%%%%%%%%%%%%%%%%%%%%%%%%%%%%%%%%%%%%%%%%%%%%%%%%%%%%%%%%%%%%
\subsubsection{Machine Learning Methods}
%
%\todo{check out SVM and KRR}
%
\Gls{ml} is at the base of most \gls{ai}s.
It is an umbrella term for programs with instructions to learn from data, i.e. gain knowledge, categorise, predict and make decisions based on data. 
%
There is a platitude of different machine learning methods: 
they can be divided into supervised (training set is labeled) and unsupervised (exploratory).  
An orthogonal division can be made by regression (continuous data) versus classification (discrete, categorical data). 
%Two architecture which don't quite fit into this classification are Generative Adversarial Network (GAN) and jk
Independent from these $2\times2$ categories there are multiple ways to let machines learn from data.
%
\Gls{nn} (one of the most popular architectures for big data\cite{Chiroma2019}) are loosely modeled after the brain\cite{bishop1994neural}.
Artificial neurons (also called nodes), which are arranged in layers, 
are connected to each of the neurons of previous and next layers
and the weights (parameters of intensity), with which the data is routed from one neuron to another, 
are optimised during training. 
%
Convolutional-layer \gls{nn} excel in picture recognition\cite{Lecun1995conv} and in quantum mechanics too\cite{westermayr2020combining}.
Other common architectures include linear regression, kernel ridge regression and support vector regression.
There are also lesser know algorithms like 
evolutionary algorithms (e.g. \gls{ga} and \gls{pso}).
\gls{pso} algorithms take advantage of the ability to cope with local optima by evolving several candidate solutions simulta- neously\cite{Villanova2010}.
%
One advantage of evolutionary algorithms is that they start with a small data set
and periodically request new data in order to solve the problem iteratively.
%\td{This is exactly what I needed because every experiment is very time intensive.} 

A genetic algorithm (GA) is a search algorithm that uses principles of natural selection and genetics to optimize a search space. A GA starts with a population of randomly generated solutions, or chromosomes, and then proceeds to breed them together to create new solutions. The new solutions are then tested for fitness, and the best solutions are selected to create the next generation of chromosomes. This process is repeated until a satisfactory solution is found.\footnote{This paragraph was written by GPT\cite{Liu2021gpt} given the input "Introduction to genetic algorithms: "}

\iffalse
\Gls{ga} uses a starting population of size $p$ ($p \in$ \td{N$^+$}) where each experiment (or data point) 
is represented by a fixed size genome of 0's and 1's in most cases. 
Each individual is then given a fitness value. 
New genomes are added and discarded from the population using 
selection, mutation and crossover operations.
%The best n ($n \in$ \td{N}) will be selected to produce offspring, 
%whose genome is a combination of both genomes with potential mutations 
%and a (mostly) random c
\fi

\Gls{pso} also uses a starting population of particles where each experiment (particle) 
is represented by its independent and dependent variables. 
It was originally inspired by the behaviour of bird flocks and fish schools\cite{Kennedy1995}.
Each particle has an associated position and velocity. 
Every movement across the search space is stochastic and influenced by its particle velocity and position as well as its and the swarm's best known position.


%%%%%%%%%%%%%%%%%%%%%%%%%%%%%%%%%%%%%%%%%%%%%%%%%%%%%%%%%%%%%%%%%%%%%%%%%%%%%%%%%%%%%%%%%
\subsubsection{EMMA}
\Gls{emma} and \gls{mars}


%%%%%%%%%%%%%%%%%%%%%%%%%%%%%%%%%%%%%%%%%%%%%%%%%%%%%%%%%%%%%%%%%%%%%%%%%%%%%%%%%%%%%%%%%
\subsubsection{Design of Experiment} %DOE
%\epigraph{"The real purpose of experiment design is to maximize the information content of the data within the limits imposed by the given constraints."}{Grahem C. Goodwin\cite{goodwin1977experiment}}
%
\begin{quote}
	{"The real purpose of experiment design is to maximize the information content of the data within the limits imposed by the given constraints."}
	- {Grahem C. Goodwin\cite{goodwin1977experiment}}
\end{quote}
%
In two cases a deliberate \gls{doe} is beneficial.
If the query of a new data point is very expensive, it is favourable to actively chose the query (e.g. drilling for oil or quantum chemical calculations). 
If the query space is so vast, that randomly querying might explore domains which might lead to uninteresting or even misleading information.
%Both encompass nearly all cases. 
% 
At the beginning of any experiment its constraints must be determined. 
%
Constraints for a given experiment include range of input and output variables as well as total time available and total number of samples/experiments that can be taken.\cite{goodwin1977experiment}
%

%FULL FACTORIAL
A naive approach to an experiment design is the full factorial \gls{doe}.
Each possible combination of discrete values is tested. 
While this is the most informative design it will more often than not be infeasible due the curse of dimensionality\cite{cherkassky1998learning}.
% 2-LVL FACTORIAL
The 2-level factorial design provides an alternative with $2^d$ (where $d$ is the number of independent variables) experiments. 
Drawbacks of 2-level factorial \gls{doe}s include no data about the inside of the search space and infeasibility for high dimensional problems.
In a full factorial or 2-level factorial design most experiments are redundant and most resources will be spent exploring high-order interaction effects\cite{gunst2009fractional}, which are often minimal to non-existent.
% RANDOM 
In order to overcome these obstacles a certain number of experiments can be chosen randomly from the search space. 
%
%A straight forward design can be to randomly chose a certain number of experiments from the search space. 
%Alternatively, a subset of from the 2-level factorial design can be chosen, which is called a fractional factorial design. 
When a subset is chosen from the factorial design, it is called a fractional factorial design. 
% PLACKETT_BURMAN
%, a fractional factorial \gls{doe}. 
%A straight forward design can be to randomly chose a certain number of experiments from the search space. 
%
%
The Plackett-Burman\cite{vanaja2007design,miller2001using,wang1995hidden} design is a special case of 2-level fractional factorial design, 
where the number of needed experiments $n$ is $n<d+4$ 
(more precisely $n=(\lfloor d\div4\rfloor+1)\cdot4$, where $\lfloor x\rfloor$ denotes the floor function on $x$) 
and each combination of levels for any pair of factors appears the same number of times. 
% HAMMERSLEY 
A drawback of 2-level factorial and random fractional designs is that the sample set is likely not evenly distributed across the search space\cite{viana2016tutorial}. 
The Hammersley design\cite{viana2016tutorial,diwekar1997efficient} is based on the Hammersley sequence and produces space filling data points. 
% LATIN HYPER CUBE
The Latin hypercube \gls{doe}\cite{viana2016tutorial,diwekar1997efficient} is a type of orthogonal \gls{doe}, 
which has the advantage that each level for each variable will be tested only once. 
A Latin hypercube \gls{doe} can be also created such that data points more uniformly distribute over the search space. 
Latin hypercube \gls{doe}s are mainly used in computer simulations which are purely deterministic and therefore have no error. 
%"Our ability to design a good experiment should depend upon our prior knowledge regarding the nature of the data generating mechanism."\cite{goodwin1977experiment}
%
%Continuous state spaces must be accommodated by arbitrary discretization.
%\cite{cohn1996neural}
\iffalse
%
\td{For any 2-level factorial DOE applies: "The effect of any factor main effect (A, B, C) or interaction (AB, AC, BC, ABC) is the difference of two averages, the average of the responses correpsonding to +1 levels and the average of the responses corresponding to -1 levels. " - from Gunst2009\cite{gunst2009fractional}}

DOEs: randomization, latin square, orthogonal experiment design, full/2-lvl factorial design, placket burman, \\
\url{https://doi.org/10.1111/j.2517-6161.1973.tb00944.x}\cite{whittle1973some}  \\
maybe \url{https://www.sciencedirect.com/science/article/pii/S0167715212003343?via\%3Dihub}
\fi

%%%%%%%%%%%%%%%%%%%%%%%%%%%%%%%%%%%%%%%%%%%%%%%%%%%%%%%%%%%%%%%%%%%%%%%%%%%%%%%%%%%%%%%%%
\subsubsection{Analysis of Variance} % ANOVA
\Gls{anova} analyses the means by means of mincing variances.
\td{at the basis of anova lies the F-test.}
%of of the continuous outputs where inputs are categories. 
The inputs of the model should be categorical and the outputs continuous. 
The statistical significance of the experiment is determined by a ratio of two variances. 
%This ratio is independent of several possible alterations to the experimental observations: 
%Adding a constant to all observations does not alter significance. 
%Multiplying all observations by a constant does not alter significance. 
\Gls{anova} statistical significance results are independent of constant bias and 
scaling errors as well as the units used in expressing observations. 
\href{https://en.wikipedia.org/wiki/Analysis_of_variance#Summary_of_assumptions}{(click here)}
Addiitonal assumptions of \gls{anova}: groups and levels should be independent, 
residual error should follow normal distribution, variance of groups should be equal. 
\href{https://statsandr.com/blog/anova-in-r/}{(click here)}

\textbf{Links}
\url{https://www.scribbr.com/statistics/anova-in-r/} just make input categorical\\
\url{https://pypi.org/project/pynova/}\\


\iffalse
\textbf{Assumptions of ANOVA}
from \url{https://statsandr.com/blog/anova-in-r/}
\begin{itemize}
    \item variable type: continuous dependent variable and categorial qualitative independent variable (also called treatments or levels). The input variables are discretized an can therefore be seen as categorial.
    \item Independence: groups and levels should be independent
    \item Normality: the residual (that is the durchschnittliche dependance of uncontrollable variables ,like room temperature or humidity) should approximately follow a normal distribution. 
    \item Equality of variance: variance in differenct groups should be equal (compare with homoscedasticity).
    \item No outliers
\end{itemize}
\fi


