%\url{https://link.springer.com/content/pdf/10.1186/2228-5326-3-8.pdf}
\td{https://en.wikipedia.org/wiki/Thesis}
\ds{what is it? What is it used for? How does it work? What different kinds are there? }
\ds{what? how? why?}
\ds{
\todo{Structure: Free writing} 
Ueber das Material, also PV und speziell CIGS (copper indium gallium sulfide) 
ueber den MC prozess berichten, also DB, i-v curven, sputtering, sem, ir, xrd
ueber den it ml stuff und statistisch, also ML algemein, pso im speziellen 
perfekt fuer intro einfach mal erwaehnen was wo in diesem capitel zu finden sein wird. 
Books PV, analytic otto? 
}
This chapter can be broken down into three section. 
The first chapter tries to shine light on the evolution of PV and give some background on \gls{cigs}.
The second chapter reads about material-scientific methods which were used during the practical part of this work. 
The last and third part focuses on the information technological, algorithmic and analytics methods used to optimize and predict material properties. 

\subsection{Photovoltaics}
%%%%%%%%%%%%%%%%%%%%%%%%%%%%%%%%%%%%%%%%%%%%%%%%%%%%%%%%%%%%%%%%%%%%%%%%%%%%%%%%%%%%%%%%%
\subsubsection{Problems of current energy supply}
The world wide energy consumption has more than doubled between 1970 and 2015\cite{BP2017} 
and according to recent studies both fossil\cite{BGR2017} and uranium sources\cite{Uran2006} 
will be exhausted within the next 100 years. 
%This time period must of course taken with a grain of salt as it can happen that even though the resource are being exhausted they can rise from one year to another as new reserves are being discovered and explored. 
Even though this time period is not exact and highly dependent on detection methods, 
this number is rather small and brings us in zugzwang to develop sustainable energy sources. 
One viable options is \gls{pv}.

\iffalse
The basis for all pvs is the photovoltaik effect which was entdeckt by Albert Einstein adn for which he got the Nobel price. 
The Prinziple is easy: When the energy (E = hv) of the light is \td{large,strong,high}er 
than the binding energy of an electron the electron is ejected with the remaining energy as kinetic energy 
\begin{math}
	E_{kin}=hv - Eb
\end{math}
Different Materials have different binding energies. 
Metals do have in general lower binding energies than covalent bound material and semiconductors do have even lowers E_b. Really? 
That's Silica is in a lot of PVs. 
The next generation of PVs. 
CIGS has in contrast to silicon based PV a direct band gap\td{source and what does that mean?}
duennschicht pv, haben eine effeftivitaet von 7-16\% vs 15-22\% \cite{Mertens2018}
\fi
\subsubsection{History of Photovoltaics}
%aenderung von widestand vonhalbleiter selen von bit ing willouhgby smith und ass joseph may fuer pv relevanten innere photoeket
%1876 william adam und richard Day zeigten mit Selenstab mit platin elektroden dass festkörper Lichenergie direkt in elekt energie umwalndeln 
%1905 erklaertle die physiklischen hinterdr'ünde mit seinem lichtquantentheorie einstein 
%die ersten solarzellen wurden in den 1950s in Nieschen anwendungen wie der Raumfahrt mit einem wirkungsgrad unter 10 prozent 
%durch die oelkriese 1973 stieg das interesse an pv und alternativen energiequellen und 1977 das erste salor...? 
The photoelectric effect was first described in 1839 by french scientist Alexandre Edmond Becquerel\cite{becquerel1839memoire}, the father of Henri Becquerel. 
Another relevant piece of the photovoltaic jigsaw was discovered 
with the discovery of photo conductivity of selenium
by British engineer Willouhgby Smith\cite{Smith1873Selenium}.
In 1876 William Adams and Richard Day\cite{Adams1876Selenium} showed that 
the energy of light can be directly converted into electrical energy by a bar of 
selenium with attached platinum electrodes.
And finally, in 1905 Einstein described the physical background of the photoelectric 
effect with his light quantum theory\cite{einstein1905erzeugung}.
In the 1950s the first solar cells (with efficiencies under 10 percent) were used in niche applications such as space flight. 
Eventually, the interest in photovoltaic and other alternative energy sources 
rose - fuelled by the oil crisis in 1973 - 
and the development of photovoltaic for the consumer market was boosted. \\

\subsubsection{Crystalline Silicon Photovoltaics}
The first marketable \gls{pv} were crystalline silicon photovoltaic modules which still have the biggest market share in the \gls{pv} segment including polycrystalline and monocrystalline silicon.
\td{what is it? What is it used for? How does it work? What different kinds are there? }

\subsubsection{CIGS}
%%%
%Materials from the chalcopyrite group (tetragonal crystal system) (eg CuInGaSe(CIGS)) can be used as absorber in such thin film cells.
CIGS ($\text{CuIn}_\text{x}\text{Ga}_{\text{(1-x)}}\text{Se}_2$) is of the chalcopyrite group (tetragonal crystal system) and can be used as thin film \gls{pv}. 
Just like CdTe, GaAs and amorphous silicon CIGS have much higher absorption coefficients 
(lower penetration depth) of visible light than crystalline silicon (see table \ref{tab:cigs:alpha}). 
These thin film \gls{pv}s not only use less material, but also can be used in flexible applications. 
%%%
%Amorphous silicon and copper indium selenide (CIS) \td{(see next chapter)} have much higher absorption coefficients of visible light than crystalline silicon (see table \ref{tab:cigs:alpha}) and can therefore be much thinner and use less material. 
%Another advantage of thin film cells is that they can be used in flexible applications. 
%which because of their high absorption coefficient of light %and therefore shallow penetration depth of light%need less material and can be used to produce flexible cells \\%of amorph silicon or CIS (see table p84) 


\begin{table}[htb]
	\small
    \center
    \begin{tabular}{cccccc}
        \hline
        \hline
		Material&   Type&    Band Gap [\ev{}]&    Wavelength [\nm{}]&    Absorption coef. $\alpha$    &Penetration Depth [\um{}]\\
        \hline
        c-Si&   indirect&   1.12&   600&    4000/cm&    2.5\\
        c-Si&   indirect&   1.12&   1000&    64/cm&    150\\
        c-Si&   indirect&   1.12&   1100&    3.5/cm&    290\\
        a-Si&   direct&      1.7&    600&    40,000/cm&  0.25\\
        CdTe&   direct&      1.45&    600&    37,000/cm&  0.3\\
		GaAs&   direct&      1.42&    600&    40,000/cm&  0.2\\
        \hline
        \hline
    \end{tabular}
	\caption{data from \cite{mertens2015photovoltaik}}
	\label{tab:cigs:alpha}
\end{table}

Amazingly, the band gap of CIGS can be varied by between 1eV and 1.7eV by varying the indium gallium ratio.
This is a result of the large difference of band gaps of CuInSe2 and CuGaSe2 (see table \ref{tab:cigs}). 
%Durch den hohen Unterschied der Bandabstaedne von CuInSe2 und CuGaSe2 kamnn durch entsprechende Mischverhaeltnisse von In und Ga der Bandabstand eingestellt werden (Tablep 142) Abbildung Bild 5.22 p143) beschreiben und bild für Modul 
%Because of the large difference of band gaps of CuInSe2 and CuGaSe2 the band gap can be configured by mixing ratio of inidum and gallium (see table \ref{tab:cigs}).

\td{Bild: electrode + n-ZnO, [2.42eV] n-CdS 40nm, [1.15eV] p-CIGS 1.5um, Molybden 0.5 um, glass substrate, borrow book} 
The standard substrate is glass 
because of it's high thermal stability, resistance and hardness. 
A flexible substrate is needed for a flexible module, though. 
Plastics have a very low melting points and metals are conducting, but can be coated with a non conducting material such as \gls{zro}.
\begin{table}[htb]
    \center
    \begin{tabular}{cccc}
        \hline\hline
        Empirical Formula&    Name&   Band Gap&    Abbreviation\\
        \hline
		\ch{CuInSe2}&       copper indium di selenide&  1&  CISe\\
		\ch{CuInS2}&        copper indium di sulfide&  1.5&  CIS\\
		\ch{CuGaSe2}&       copper gallium di selenide&  1.7&  CIGSe\\
		\ch{CuGaS2}&        copper gallium di sulfide&  1.55&  CIGS\\
        \hline\hline
    \end{tabular}
	\caption{band gaps of different chalcopyrites}
	\label{tab:cigs}
\end{table}

%%%%%%%%%%%%%%%%%%%%%%%%%%%%%%%%%%%%%%%%%%%%%%%%%%%%%%%%%%%%%%%%%%%%%%%%%%%%%%%%%%%%%%%%%%%%%
%%%%%%%%%%%%%%%%%%%%%%%%%%%%%%%%%%%%%%%%%%%%%%%%%%%%%%%%%%%%%%%%%%%%%%%%%%%%%%%%%%%%%%%%%%%%%
\subsection{Materials and Their Analysis}
\subsubsection{Zirconium oxide}
Zirconium oxide \gls{zro} is a ceramic with a band gap of 5-7 eV and dielectric constant of 15-22 at room temperature\cite{Anwar2017}. 
This makes it attractive as an insulator for semiconductor and \gls{pv} industry. 
It is monoclinic below 1050 °C, tetragonal between 1170 °C and 2370 °C, and cubic above 2370 °C\cite{Nielsen2005}.
The cubic phase can be stabilzed down to room temperature by the addition of magnesia (\ch{MgO}), calcia (\ch{CaO}) or yttria (\ch{Y2O3}) which avoids mechanical failing due to shrinkage 
when cooling and undergoing phase transistion\cite{Nielsen2005}.
%due to phase transition\cite{Nielsen2005}.
%\td{[R. Stevens, 1986. Introduction to Zirconia. Magnesium Elektron Publication No 113]}
%Ralph Nielsen "Zirconium and Zirconium Compounds" in Ullmann's Encyclopedia of Industrial Chemistry, 2005, Wiley-VCH, Weinheim. doi:10.1002/14356007.a28_543
%It is not poisonous. \td{cite?}
It is very resistant to acids (except \ch{HF} and hot \gls{h2so4}) and alkalis\cite{Nielsen2005}.
%See ref \cite{Nielsen2005} for 2.3 Hydrous Zirconium Oxide 2.3 and 2.20 Zirconium Alkoxides.
Hydrous Zirconium Oxide (\ch{Zr(OH)8 * 16 H2O}) gel can be produced by neutral hydrolysis of sodium zirconate (\ch{NaZrO3}). 
"Zirconium alkoxides hyrdolize quite easily, [providing] a route to high purity, high-surface-area zirconium oxide"\cite{Nielsen2005}.

%\td{what is important properites of zro2? }
%\td{what is it? What is it used for? How does it work? What different kinds are there? }

\subsubsection{Sol-Gel}
\td{what is it? What is it used for? How does it work? What different kinds are there? }
One of the advantages of sol-gel process is that it can be used in roll-to-roll procedures.

%%%%%%%%%%%%%%%%%%%%%%%%%%%%%%%%%%%%%%%%%%%%%%%%%%%%%%%%%%%%%%%%%%%%%%%%%%%%%%%%%%%%%%%%%%%
\subsubsection{Sputtering}
%SPUTTERING FREEWRITING: 
%what is sputtering? \\
Sputtering is the processes of highly energetic ions hitting a surface and atoms or molecules being expelled. 
This is called a physical vapor deposition (PVD) technique. 
PVD can be divided into activation by thermal energy and activation by energetic particle bombardment. 
Sputtering is of the latter, which 
is advantageous if substrates can't withstand high temperatures.
%\td{what can it be used for?}\\
Sputtering evolved from being a curious experiment in the middle of the 20th century to having various applications in research and engineering.
Use cases vary from thin films depositions for \gls{pv}, for electrical circuits or for storage media such as CDs and DVDs 
over sputter cleaning and etching to analysis.
Advantages of 
sputtered thin films include good adhesion to the substrate and good step coverage. 

%\td{how does it work?}\\
A high voltage is applied to 
two parallel electrode with low pressure gas in between. 
The target acts as cathode and the substrate (holder) as anode (see \td{fig}).
%the target which acts as a cathode with the substrate as anode in a parallel geometry (see \td{fig}).
The cathode, then, emits electrons which collide with a gas particle (mostly argon because of it's inert properties and potential to transfer more kinetic than lighter noble gases). 
Some gas particles may get ionized by the collision and the gas cation is accelerated to the cathode. 
If the cation has enough energy it will bump off a target atom or molecule from the surface. 
This happens by a cascade of momentum transfers, which can reach the surface again (see fig \td{cp sigmund69}). 
If the surface particle gets momentum pointing away from the bulk and the remaining kinetic energy is higher then the binding energy, the particle is sputtered. 
The pressure should be small, such that the sputtered particle has a long \gls{mfp}, but on the other hand there is a minimum pressure to keep the plasma going. 
Usual pressures are around 1 Pa ($10^{-2}$ mbar) or lower\cite{Swann1988}.

A magnetron can be placed behind the cathode (target) in order to trap ejected electrons close to the source. 
This prevents high energy electrons from reaching the target and undoing the deposited layer and this also increases the probability of an electron colliding with an argon atom and ionizing it.

When oxygen or nitrogen are added to the  gas this is called reactive sputtering.
Sputtered atoms will react with the gas and result in oxide or nitrides layers, respectively.
The stoichiometry of the resulting layer can be regulated by gas ratios, but too much reactive gas can lead to target poisoning\td{cite? What is poisoned?}.

The limitation of only being able to use conducting materials as targets can be circumvented by using a radio frequency electrical field. 
This prevents a charge building up on the target. \td{why is a charge?}
Although RF sputtering is more versatile, DC sputtering is more common because of it's simpler system and economical reasons.
\td{sputtered particles are neutral and not influenced by the electrical field}

\subsubsection{Scanning electron Microscopy}
\url{https://doi.org/10.1016/B978-0-12-816806-6.00017-0}\\
\Gls{sem} is a microscopical technique which allows visualisation of surfaces with features in the nano meter regime. 
While optical microscopes use visible light and optical lenses the \gls{sem} uses accelerated electron beams and electrostatic and electromagnetic lenses.
%Instead of light, it uses electrons with shorter wavelengths. zum abbilden
This allows the generation of much more detailed images due to the shorter wavelengths of electrons compared to light\cite{Kaliva2020}.
The electron beam produces X-rays, elastically backscattered (primary) electrons, inelastic (secondary) electrons and Auger electrons. 
Secondary electrons carry information to conclude morphology and topology of the sample while X-rays can be used to identify the elements. 
Electrons are \td{created} by either \gls{feg}, where are strong electrical field rips electrons from the bulk and thermionic guns where the filament (tungsten W or \ch{LaB6} (brigther and longer lasting but more expensive)) is heated until electrons are emitted. 
Electrons are then accelerated by a voltage of 2 to 40 kV and bundled into narrow beams\cite{Vernon2000} by lenses.
\td{A high \gls{mfp} is needed for electrons to travel from the source to the sample and to the detector. Thus, a very low pressure is in the inside.}
It was used as a preliminary way of checking the surface condition. 
%This allows greater detail. 
why? 
\td{what? how? why?}
what is sputtering, 
how does it work, 
what different kinds of sputtering is there, 
what is is used for? 
\cite{McMullan1995}
\cite{Vernon2000}
\cite{Kaliva2020}


\subsubsection{Infrared absorption}
\td{IR around page 45;
REFLECTIVITY? can also be used for determination of thickness p47
}
\subsubsection{X-Ray Diffraction}
\gls{xrd} is used to study the crystalline strucutre of materlias since X-rays wavelengths (between 0.2 and \nm{10}) are comparable to the interatomic spacing of crystalline solids.  \cite{Kaliva2020}.
Each crystalline material has discreet atomc structure, which upon irradiation with X-rays causes constructive and destructive interference and generates unique diffraction patterns. 
XRD \td{spectra} of crystalline materials feature distinct peaks, whereas amorphous materials exhibit a broad cruve with a maximum extending over several degrees (2$\theta$).

\url{https://chem.libretexts.org/Courses/Franklin_and_Marshall_College/Introduction_to_Materials_Characterization__CHM_412_Collaborative_Text/Diffraction_Techniques/X-ray_diffraction_(XRD)_basics_and_application}\\


\subsection{Machine Learning}
\paragraph{History}
some text to test
\paragraph{different kinds} 
supervised vs unsupervised
classification (discrete) vs regression
\paragraph{Particle Swarm Optimization}
\paragraph{Princlipal Component Analysis}
reinforcement learning could also have been a nice option
\paragraph{Linear Regression}
