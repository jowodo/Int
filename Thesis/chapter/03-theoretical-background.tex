%\url{https://link.springer.com/content/pdf/10.1186/2228-5326-3-8.pdf}
\todo{Structure: Free writing} 
Ueber das Material, also PV und speziell CIGS (copper indium gallium sulfide) 
ueber den MC prozess berichten, also DB, i-v curven, sputtering, sem, ir, xrd
ueber den it ml stuff und statistisch, also ML algemein, pso im speziellen 
perfekt fuer intro einfach mal erwaehnen was wo in diesem capitel zu finden sein wird. 
Books PV, analytic otto? 
\subsection{Photovoltaic and CSGI}
\iffalse
The grundlage for all pvs is the photovoltaik effect which was entdeckt by Albert Einstein adn for which he got the nobel price. 
The Prinziple is easy: When the energy (E = hv) of the light is \td{large,strong,big,high}er than the binding energy of an electron the electron is ejected with the remaining energy as kinetic energy 
\begin{math}
	E_{kin}=hv - Eb
\end{math}
Different Materials have different binding energies. Metalls do have in general lower binding energies than covalent bound material and halbleiter do have even lowers E_b. 
Thats Silica is in a lot of PVs. 
The next generation of PVs. 
CIGS has in contrast to silicium based PV a direct band gap\td{source and what does that mean?}
duennschicht pv, haben eine effeftivitaet von 7-16\% vs 15-22\% \cite{Mertens2018}
\fi
%%%%%%%%%%%%%%%%%%%%%%%%%%%%%%%%%%%%%%%%%%%%%%%%%%%%%%%%%%%%%%%%%%%%%%%%%%%%%%%%%%%%%%%%%
\subsubsection{Problems of current energy supply}
The world wide energy consumption has more than doubled between 1970 and 2015\cite{BP2017} and according to recent studies both fossil\cite{BGR2017} and uranium sources\cite{Uran2006} 
will be exhausted within the next 100 years. 
%This time period must of course taken with a grain of salt as it can happen that even though the resource are being exhausted they can rise from one year to another as new reserves are being discovered and explored. 
Even though this time period is not exact and highly dependent on external factors, this number is rather small and brings us in zugzwang

\subsubsection{History of Photovoltaics}
%aenderung von widestand vonhalbleiter selen von bit ing willouhgby smith und ass joseph may fuer pv relevanten innere photoeket
The photoelectric effect was first decribed in 1839 by french scientist Alexandre Edmond Becquerel\cite{becquerel1839memoire}, the father of Henri Becquerel. 
%1876 william adam und richard Day zeigten mit Selenstab mit platin elektroden dass festkörper Lichenergie direkt in elekt energie umwalndeln 
With the discovery of the correlation between the {change of the }resistance and the irradiating light by british engineer Willouhgby Smith and his assistant Joseph May another relevant piece of the PV jigsaw was discovered. 
In 1876 William Adam and Richard Day showed that with a bar of selenium attached with platin electrodes the energy of light can be directly converted into electrical energy.
%1905 erklaertle die physiklischen hinterdr'ünde mit seinem lichtquantentheorie einstein 
And finally, in 1905 Einstein described the physical background of the photoelectric effect with his light quantum theory\cite{einstein1905erzeugung}.

%die ersten solarzellen wurden in den 1950s in Nieschen anwendungen wie der Raumfahrt mit einem wirkungsgrad unter 10 prozent 
The first solar cells were used in the 1950s in niche applications such as the space flight with efficiencies under 10 percent. 
%durch die oelkriese 1973 stieg das interesse an pv und alternativen energiequellen und 1977 das erste salor...? 
Though, fuelled by the oil crisis in 1973 the interest in photovoltaic and alternative energy sources in general rose and the development of mass product was boosted. \\
consumer market
\\
K5\\
%Neben der g"ngigen kristallienen Siliziem Zelle gibt es auch Duennnschichtzellen wegen dem hohen Absorptionskoeffizenten und der geringen Eindringtiefe von veispielsweise amorphem Silizium oder auch CIS (siehe table p 84) wird nur wenig Material benötigt und flexible Zellen konnen hergestellt werden 
Besides the common crystalline silicon cells there are thin film cells which because of their high absorption coefficient of light and therefore shallow penetration depth of light
need less material and can be used to create/produce flexible cells \\
of amorph silicon or CIS (see table p84) 

\begin{table}
    \begin{tabular}{cccccc}
        \hline
        \hline
        Material&   Art&    Bandluecke [eV]&    Wavelenght [nm]&    Absorptionskoeffizent alpha&    Eindringtiefe [um]\\
        \hline
        c-Si&   indirect&   1.12&   600&    4000/cm&    2.5\\
        c-Si&   indirect&   1.12&   1000&    64/cm&    150\\
        c-Si&   indirect&   1.12&   1100&    3.5/cm&    290\\
        a-Si&   direct&      1.7&    600&    40,000/cm&  0.25\\
        CdTe&   direct&      1.45&    600&    37,000/cm&  0.3\\
        a-Si&   direct&      1.42&    600&    40,000/cm&  0.2\\
        \hline
        \hline

    \end{tabular}
\end{table}

CIGS\\ 
%Materialien aus der Grupper der Chalkopyrite wisen allesamt eine XXX Gitterstruktur auf genauso wie Namengeber der Gruppe, das Chalkopyrit (CuFeS2). 
Materials from the chalcopyrite group (tetranogal crystal system) (eg CuInGaSe) can be used as absorber in such thin film cells.
Meist wird ein Material mit der Summenformel $CuIn_XGA_{(1-x)}Se_2$ verwendet. 
Durch den hohen Unterschied der Bandabstaedne von CuInSe2 und CuGaSe2 kamnn durch entsprechende Mischverhaeltnisse von In und Ga der Bandabstand eingestellt werden (Tablep 142) Abbildung Bild 5.22 p143) beschreiben und bild für Modul 

{Bild: electrode + n-ZnO, [2.42eV] n-CdS 40nm, [1.15eV] p-CIGS 1.5um, Molybden 0.5 um, glas substrat} 
\begin{table}
    \begin{tabular}{cccc}
        \hline\hline
        Materialkombination&    Name&   Bandabstand&    Kürzel\\
        \hline
        CuInSe2&        copper indium di selenide&  1&  CISe\\
        CuInS2&        copper indium di sulfide&  1.5&  CIS\\
        CuGaSe2&        copper gallium di selenide&  1.7&  CIGSe\\
        CuGaS2&        copper gallium di sulfide&  1.55&  CIGS\\
        \hline\hline
    \end{tabular}
\end{table}


Um ein Modul flexibel zu machen kann statt Glas Stahl als stbstrat verwendet werden. 
Allerdings wird ein Isolater zwischen Stahsubstrat und CIGS Zelle benötigt


\subsection{Zirconium oxide}
\subsection{Sputtering}
%https://en.wikipedia.org/wiki/Sputter_deposition
\subsection{SEM}
\url{https://doi.org/10.1016/B978-0-12-816806-6.00017-0}\\
\subsection{Infrared absorption}
\td{IR around page 45;
REFLECTIVITY? can also be used for determination of thickness p47
}
\subsection{X-Ray Diffraction}
\url{https://doi.org/10.1016/B978-0-12-816806-6.00017-0}\\
\url{https://chem.libretexts.org/Courses/Franklin_and_Marshall_College/Introduction_to_Materials_Characterization__CHM_412_Collaborative_Text/Diffraction_Techniques/X-ray_diffraction_(XRD)_basics_and_application}\\
\subsection{Particle Swarm Optimization}
\subsection{Machine Learning}
\subsubsection{History}
\subsubsection{different kinds} 
supervised vs unsupervised
classification (discrete) vs regression
\subsection{Princlipal Component Analysis}
reinforcement learning could also have been a nice option
