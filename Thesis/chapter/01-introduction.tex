%%%%%%%%%
%%%%%%%%%%%%%%%%%%%%%%%%%%%
%%%%%%%%%%%%%%%%%%%%%%%%%%%%%%%%%%%%%%%%%%%%%%%%%%%%%%
\td{describe everything what is mentioned in \ref{sec:exp}}
\Gls{pv} is one big hope when trying to become carbon neutral as it uses the energy provided by sun directly in contrast to renewable energy sources (e.g. wind and water) or even carbon based sources.
One sort of \gls{pv} are \gls{cigs} \cite{Vasekar2010} cells. % https://doi.org/10.1016/j.tsf.2009.09.033}
\td{was sind vorteile?}
In order to make a module, multiple cells are operated in series. 
The cells must be applied to a non-conducting surface.
Glass is a good non-conducting substrate, but very rigid and brittle. 
An alternative is steel, which is ductile, inexpensive and highly available, but conducting. 
An insulating layer must therefore be applied to the steel substrate before any \gls{cigs} cells can be applied.
A non-toxic material which is suitable for this application is \gls{zro}. 
An economic and scalable method is doctor blading via a \gls{sg} process. 
\gls{sg} processes often produce porous layers. 
In this work a dense, insulating and homogeneous layer is pursued. 
\Gls{ml} can help to uncover complex non-linear relations, such as the influence of the 
production factors on the thickness and resistance of the resulting layer.
The optimisation of the conductance if performed with a particle swarm optimization 
algorithm. 
The resulting optimum is compared with other optimisation methods.
