%%%%%%%%%
%%%%%%%%%%%%%%%%%%%%%%%%%%%
%%%%%%%%%%%%%%%%%%%%%%%%%%%%%%%%%%%%%%%%%%%%%%%%%%%%%%
\Gls{pv} is one viable option when becoming carbon neutral. 
Furthermore, 
it uses the suns energy directly in contrast to other energy sources (e.g. wind, water or even carbon based) and therefore 
it is fit to be used in futuristic Dyson spheres\cite{dyson1960search} which harness the power output of the whole sun.
%
One sort of \gls{pv} are \gls{cigs}\cite{Vasekar2010} cells. 
Due to their large absorption coefficient, less material is needed and they can be made thinner and flexible. 
%\td{was sind vorteile?}
In order to make a module, multiple cells are operated in series. 
The cells must be applied to a non-conducting surface.
Glass is a good non-conducting substrate, but very rigid and brittle. 
An alternative is steel, which is ductile, inexpensive and highly available, but conducting. 
An insulating layer must therefore be applied to the steel substrate before any \gls{cigs} cells can be \td{applied}.
A non-toxic material which is suitable for this application is \gls{zro}. 
An economic and scalable method is doctor blading via a \gls{sg} process. 
\gls{sg} processes often produce porous layers. 
In this work a dense, insulating and homogeneous layer is pursued. 
\Gls{ml} can help to uncover complex non-linear relations, such as the \td{influence} of the 
production factors on the thickness and resistance of the resulting layer.
The minimization of the conductance is performed with a particle swarm optimization 
algorithm. 
The resulting optimum is compared with other optimisation methods.

\td{
This work used a rather unconventional approach. 
Firstly, a problem was posed. Then, some minor literature research was done and the problem was tackled. 
Finally, the engineering problem was solved within the limits of acceptability, but not aussreichend for the author. 
The larger bulk of literature research was done after the practical work has been finished in order to solve the secondary scientific problem of predicting outputs and approximate dependencies. 
}

\td{The remainder of this work is organized with section 2 (section number auto?) providing some background on ...}
