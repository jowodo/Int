\documentclass[a4paper]{article}
\usepackage[margin=3cm]{geometry}
\usepackage{xcolor}
\usepackage{graphicx}
\usepackage{lipsum}
\usepackage{amsmath}
\usepackage{pdfpages}
\usepackage{url}
\usepackage{hyperref}
\hypersetup{
	colorlinks = true,
	linkbordercolor = {white},
	urlcolor = {blue},
	linkcolor = {blue},
	citecolor = {blue}
}
\usepackage{acronym}
\usepackage[acronym,nomain]{glossaries}
\makeglossaries
\newcommand{\todo}[1]{\textbf{\textcolor{red}{#1}}}
\newcommand{\td}[1]{\textbf{\textcolor{red}{#1}}}

\title{Functional oxide layers for electrical isolation and chemical passivation of steel substrates}
\author{Johann Dorn}

\begin{document}
\maketitle
\iffalse
my notes
\fi
%\begin{acronym}
	\newacronym{nemo}{amat}{dolorem}
	\newacronym{zrpro}{Zr(PrO)$_4$}{zirconium(IV)propoxide}
	\newacronym{acac}{Hacac}{acetylacetone}
	\newacronym{acoh}{AcOH}{acetic acid}
%\end{acronym}

\printglossaries
\section{Preface}
some 
\Ac{abktag}
\gls{nemo}
worlds
\Ac{abktag}
adsf

\section{Introduction}
\section{Aims and Objectives}
\section{Experimental}
\subsection{Substrate Preparation}
\subsection{Cutting of the steel foil}
\subsection{Solution Preperation}
\label{sec:cut}
There is a red foil cutter in the vacuum room, which cuts the foil without much bending.
Alternatively, the foil can be cut without bending by cutting repeatedly with a cutter knife.
The foil is cut into 2.5 x 2.7 mm plates.
The small plates are marked with an diamond cutter pen.
The plates are cleaned with 1ml of Hellmanex III in 50-100 ml dion. water in the sonic bath for 15 min, then in dion. water for 15 min and finally in isopropanol for 15 min. 
The samples are dry blown with dry N$_2$ and stored until doctor blading.

\subsection{Solution}
Two main recipes were used and the ingridients varied. 
The first recipe was adopted from Ref. \cite{Anwar2017} was based on \gls{zrpro} in 
\subsubsection{Aquatic solution}
\subsubsection{Buthanolic solution}
\label{sec:sol}
The recipe for the solution was adapted from ref \cite{Hu2016}.
The standard concentration will be described first and then the differences of higher concentrated solutions:
4.9 ml of 1-buthanol (BuOH) are put into a beaker glass (or similar, preferably with cap) with a stirrer. 
0.1 ml of zirconium(IV)propoxide solution (Zr(PrO)$_4$) are added while stirring.  
After 10 to 15 minutes 0.05 ml (approximately one mole equivalent of Zr(PrO)$_4$) acetylactate (AcAC) is added and stirred for another 10 to 15 minutes. 
Finally, 1 ml of isopropanol (IPO) is added to the mixture and stirred for additional 20-30 minutes. 
Following stirring times (in minutes) were tested and didn't have an influence on stability of the solution: 10-10-20, 10-10-45, 30-30-180. 
In order to make a double concentrated solution, the volume of Zr(PrO)$_4$ and AcAc is doubled and BoOH is decreased by the volume of Zr(PrO)$_4$. 
The real concentration is not double of the original, though, but rather 1.7 fold because volume of IPO is kept constant.

\begin{table}[h]
	\centering
	\begin{tabular}{clllll}
		\hline
				&1F		&2F		&3F		&4F		&5F		\\
		\hline
		conc. [a.u.]	&1		&1.7	&2.6	&3.5	&4.4	\\
		\hline
		BuOH [ml]		&4.95	&4.9	&4.85	&4.8	&4.75	\\
		Zr(OPr) [ml]	&0.05	&0.1	&0.15	&0.2	&0.25	\\
		AcAc [ml]		&0.0125	&0.025	&0.0375	&0.05	&0.0625	\\
		IPO [ml]		&2		&2		&2		&2		&2		\\
		\hline
	\end{tabular}
\end{table}

\subsection{Doctor blading}
\label{sec:DB}
The temperature of the heating plate is set to 200 $^o$C.
The temperature of the vacuum plate is set and waited until reached.
The sample is placed on the vacuum plate and tested if it can be held by the under pressure.
The velocity is set and a mini test run is performed. 
The blade is put in position. 
100-125 $\mu$l of solution is applied with an 10-1000 $\mu$l pipette and the doctor blading is started immediately. 
After evaporation of the solution, the vacuum is turned of, the 'blade pusher' put into initial position, the blade removed and excess solution removed with a wipe. 
The small metal plate is transferred to the hot heating plate and rests on there for 3-5 min. 
The process is repeated as wished. 

\section{Evaluation and Computational Details}
\subsection{Evaluation of Samples}
\label{sec:eval}
For every I-V curve (aluminium dot) the gradient $g$ at V=0 is calculated by taking 5 points after the origin and 5 points before the origin, averaging their V and I values and calculating i
\begin{equation}
	g = \frac{I_{n+1} - I_n}{V_{n+1} - V_n}.
\end{equation}
As a measure of conductance a distance D from an ideal non-conducting case. The average of the negative base 10 logarithm subtracted from an ideal non-conducting gradient of $10^{-13}$ 
\begin{equation}
	D = \sum_i^N \frac{ -log_{10}(g_i) - 13}{N}
	\label{eq:D}
\end{equation}
Another measure is the density of shorted species $\rho_{s}$ is calculated in following way:
\begin{equation}
	s_i = \begin{cases}
	1 &\text{if} \quad -log(g_i) < 5 \\
	0 &\text{if} \quad -log(g_i) \geq 5 \\
	\end{cases}
\end{equation}
\begin{equation}
	\rho_s = \sum_i^N \frac{s_i}{N}
	\label{eq:rho}
\end{equation}
Other estimates of the conductance are the averages:
\begin{equation}
	G_1 = log \left( \sum_i^N \frac{g_i}{N} \right)
\end{equation}

\begin{equation}
	G_2 =  \sum_i^N \frac{log(g_i)}{N}
\end{equation}

\subsection{Sample Selection}
\label{sec:ss}
An evolutionary approach was chosen, namely a multi-objective Particle Swarm Optimization (PSO) with a multi-response
Multivariate Adaptive Regression Splines (MARS) model\cite{Villanova2010,Kennedy1995,Breiman1997,Carta2011}.
%
"PSO is a population based heuristic inspired by the flocking behavior of birds. 
To simulate the behavior of a swarm, each bird (or particle) is allowed to fly towards the optimum solution."\cite{Villanova2010}
%
Initially the input parameters (independent variables), their boundaries and number of equidistant levels for each parameter are declared (see table \ref{tab:input}).
Next, the output variables (dependant variables), their weights in the objective function (the function which should be optimized) are specified and if they should be minimized or maximized is noted.
%
%An initial population of particles, i.e. experiments with certain parameters, is chosen out of the population space (space spanned by all possible combinations of input parameters), 
\begin{table}[htb]
	\centering
	\begin{tabular}{cc cc cc}
		\hline
		Zr(PrO)$_4$ conc. [21 g/L]	&layers	&$T_{DB}$[$^o$C]	&$v_{DB}$[mm/s]	&$T_{cal}$[$^o$C]	&$v_{cal}$[$^o$C/hour]	\\
		\hline
		2				&4		&40					&10				&300				&120	\\
		3				&6		&50					&12				&400				&360	\\
		4				&8		&60					&14				&500				&600	\\
		5				&10		&70					&16				&					&840	\\
						&12		&80					&18				&					&1080	\\
						&		&					&20				&					&		\\
		\hline
	\end{tabular}
	\caption{Discrete levels of each input parameter}
	\label{tab:input}
\end{table}

The first step is to select an initial population (ensemble of experiments), which is chosen randomly from the population space. 
The samples are made, measured and evaluated according to sections \ref{sec:cut}-\ref{sec:eval} and the distance $D$ (see eq. \ref{eq:D}), $\rho_s$ (see eq. \ref{eq:rho}), $n_{layers}$ (numbers of layers) and $v_{cal}$ (heating rate of calcination process in $^o$C/min) are supplied to the program. 
The program uses this data to estimate a response for each output variable (and to choose a fraction of the initial population which is allowed to propagate).
The response variables for the entire population space is calculated. 
The current population - each of the particles independently - moves towards the optimum solution.
The population for the next time step is outputted and the experiments are again executed, measured and evaluated.


scarce data may lead to overfitting\cite{Lecun1995conv}
\section{Results and Discussion}
\section{Outlook}




Making of the solution for the sol-gel process:
For a single concentrated solution 0.05 ml of Zr(IV)Propoxide are added while stirring to 4.95 ml of Buthan-1-ol and stirred for 15 minutes. 
0.013 ml (or one molar equvilent of Zr) of Acetylacetone is added to the stirring solution. 
After another 15 minutes 1 ml of acetic acid is added and stirred for 30 minutes to stabilize the solution up to 24h. 

The concentration can be increased up to 5 times being stable for a minimum of 4 hours. 
The sol-gel process produces am homogeneous transparent crystalline zirconia oxide layer. 
homogeneity can be mainly controlled via blade velocity and temperature and layers can be stacked.

\clearpage
\bibliographystyle{ieeetr}
\bibliography{int}
\end{document}
